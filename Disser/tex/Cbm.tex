\chapter{Эксперимент CBM на FAIR}\label{sec:secCbm}

\section{Экспериментальная установка CBM}\label{sec:secCbmSetup}

\subsection{Вершинный микродетектор MVD}\label{sec:secMVD}

\subsection{Кремниевая трекинговая система STS}\label{sec:secSTS}

From CBM STS TDR:

The detector system’s task is to measure the trajectories and momenta of charged particles originating from the interactions of heavy-ion beams with nuclear targets. Up to 1000~charged particles are produced per interaction, at rates up to 10~MHz to enable CBM physics with rare observables. The track reconstruction has to be achieved with 95\%~efficiency and a momentum resolution $\Delta p / p = 1\%$. These requirements can be fulfilled with a tracking system of 8~low-mass layers of silicon microstrip sensors located at distances between 30~cm and 100~cm downstream of the target inside the magnetic dipole field. The sensors are mounted onto lightweight mechanical support ladders and read out through multi-line micro-cables with fast self-triggering electronics at the periphery of the stations where cooling lines and other infrastructure can be placed. The micro-cables will be built from sandwiched polyimide-Aluminum layers of several $10 \mu m$ thickness. The microstrip sensors will be double-sided with a stereo angle of \SI{7.5}{\degree}, a strip pitch of $58 \mu m$, strip lengths between 20~and~60~mm, and a thickness of $300 \mu m$ of silicon. According to the CBM running scenario the maximum non-ionizing dose for the sensors closest to the beam line does not exceed $10^{14}$ $n_{eq}$ $cm^{-2}$. The STS is operated in a thermal enclosure that keeps the sensors at a temperature of about \SI{-5}{\degreeCelsius}. The heat dissipated in the read-out electronics is removed by a $CO_{2}$ cooling system. The mechanical structure of the detector system including the service and signal connections is designed such that single detector ladders can be exchanged without disconnecting and removing more than one detector station.

\subsection{Детектор черенковских колец RICH}\label{sec:secRICH}

\subsection{Мюонная система MUCH}\label{sec:secMUCH}

\subsection{Детектор переходного излучения TRD}\label{sec:secTRD}

\subsection{Время-пролётный детектор TOF}\label{sec:secTOF}

\subsection{Электромагнитный калориметр ECAL}\label{sec:secECAL}

\subsection{Детектор PSD}\label{sec:secPSD}
