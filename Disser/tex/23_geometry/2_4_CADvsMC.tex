% ================================================================================================================
%   ____    _    ____                      __  __  ____ 
%  / ___|  / \  |  _ \    __   _____      |  \/  |/ ___|
% | |     / _ \ | | | |   \ \ / / __|     | |\/| | |    
% | |___ / ___ \| |_| |    \ V /\__ \_    | |  | | |___ 
%  \____/_/   \_\____/      \_/ |___(_)   |_|  |_|\____|
%                                                       
% ================================================================================================================

\section[Сравнение представления геометрии в GEANT/ROOT и САПР]{Сравнение представления геометрии\\ в GEANT/ROOT и САПР}\label{sec:secROOTvsCAD}

Разница между двумя способами описания геометрической информации в САПР и пакетах моделирования прохождения частиц через материал GEANT/ROOT заключается в двух пунктах. Во-первых, отличается способ задания геометрических форм. В САПР применяется граничное представление (BREP), для описания которого используются понятия типа <<поверхность>>, <<грань>>, <<ребро>>, <<кривая>>, и за которыми стоят соответствующие уравнения, описывающие эти объекты в пространстве. В GEANT/ROOT применяется конструктивная твердотельная геометрия (CSG), которая оперирует понятиями <<примитив>> и <<Булева операция>>. Очевидно, что и за этими объектами также стоят конкретные уравнения, описывающие кривые и поверхности, однако есть существенное различие описанное ниже. Во-вторых, отличается способ задания взаимоотношения форм в пространстве. В САПР, по аналогии с тем, как человек воспринимает окружающий мир, присутствует некоторое бесконечное окружающее пространство без материала, а все предметы находятся в этом пространстве. Невозможна такая ситуация, чтобы один объект находился внутри другого --- в таком случае подразумевается, что во втором есть соответствующая полость, освобождающая место под первый объект.
%При этом получается, что границы двух тел, стоящих рядом, совпадают, т.е происходит дублирование информации.
%\todo В силу математики это может приводить к неприятным эффектам. \todo
В GEANT/ROOT для описания взаимоотношения форм используется иерархия объёмов. Это объясняется тем, что такой метод более удобен для описания геометрии, где главной задачей является однозначное задание материала в каждой точке пространства. Вводится понятие объёма --- сущности, имеющей форму и материал. Из всех объёмов выбирается один, называемый объёмом верхнего уровня, а остальные помещены либо в него, либо в какой-то другой, формируя таким образом дерево объёмов.
%\todo При этом не происходит дублирования границ и нет упомянутых выше эффектов. \todo

Процесс проектирования современной экспериментальной установки подразумевает разнообразное компьютерное моделирование этой установки.

\textbf{В первую очередь выполняется компьютерное геометрическое моделирование в трёхмерном пространстве с целью получения конструкторской документации и анализа расположения элементов в пространстве. Геометрическая модель для этих целей обычно строится средствами систем автоматизированного проектирования (САПР), в которых стандартным способом представления геометрической информации является граничное представление (BREP).}
Также, как и в любой другой прикладной области, необходимо выполнять многочисленные расчёты, которые нередко требуют геометрическую модель в качестве входных данных.
Так, например, в инженерно-конструкторской среде широкое распространение получил метод конечных элементов (МКЭ) для решения задач прочности и устойчивости механических конструкций, динамики жидкостей и газов и т.д.
%Применяются соответствующие способы описания геометрии
%ссылка на соответствующую секцию

Отличительной особенностью сферы физики частиц является то, что в процессе проектирования установки помимо типовых расчётов требуется выполнение моделирования прохождения частиц через материал, которое чаще всего выполняется физиками, формулирующими требования к конструкции установки, но в общей массе не владеющими САПР. Такое моделирование достаточно специфично, но оно также выполняется над геометрической моделью, в идеальной ситуации --- максимально подробной, совпадающей с полной детальной моделью, полученной инженерами-конструкторами с помощью САПР. Также стоит отметить, что процесс конструирования, в том числе получения инженерной геометрической модели, и процесс моделирования физики не имеют чётко определённого порядка и тесно между собой переплетены. В результате обоих процессов уточняются геометрические параметры деталей, компоновка узлов, применяемые материалы, и т.д. Это приводит к необходимости постоянного обмена геометрической информацией.

\textbf{Повторное обоснование использование CATIA}

Как было сказано выше, инженеры для получения геометрической модели используют САПР. Во многих физических лабораториях, включая CERN, GSI и ОИЯИ, применяется САПР CATIA~v5. Моделирование взаимодействия частиц с материалом широко применяет метод Монте-Карло (MC) и реализовано в соответствующих программных пакетах, многие из которых основаны на фреймворках GEANT4 или GEANT3, разработанных в CERN. Также часто применяют поход Virtual Monte-Carlo (VMC), в котором все процедуры, связанные с геометрией, поручены системе ROOT. Все перечисленные физические пакеты (GEANT3, GEANT4, ROOT, далее коротко GEANT/ROOT) используют представление геометрии, принципиально отличающееся от BREP. Модели для GEANT/ROOT часто называют MC-моделями. Это отличие состоит из двух пунктов, подробно описанных в~\ref{sec:secROOTvsCAD}, и приводит к невозможности прямого обмена геометрическими моделями между физиками и инженерами.

Главным фактором против прямой конвертации в том или ином направлении является то, что она имеет малую практическую пользу. Одна и та же геометрическая модель с точки зрения разных задач может быть одновременно оптимальна и, наоборот, избыточна или недостаточна. Это просто понять на следующем примере. С точки зрения инженерного проекта массив болтов, вкрученных в корпус, представляет собой важную информацию. В чертежах и другой конструкторской документации ошибка в точном положении отверстий, их диаметре, типе резьбы болтов и т.д. может привести к невозможности собрать продукт после изготовления отдельных компонентов. В то же время, в САПР принято не хранить, и следовательно не визуализировать, витки резьбы с целью снижения нагрузки на графический адаптер ЭВМ. Это значит, что резьба присутствует только формально, в документации, а геометрическая модель имеет лишь условное обозначение резьбы в соответствующем месте. С точки зрения моделирования прохождения частиц через материал в зависимости от расположения в общей установке подобные подробности могут оказаться как критическими, так и наоборот излишними и вызывающими значительное увеличение времени выполнения моделирования.
Например, форма ионопровода, который расположен в непосредственной близости к пучку, где присутствуют высокие потоки частиц, может оказать влияние на функционирование всей установки, в то время как форма корпуса детектора, имеющего размеры порядка нескольких метров и находящегося в отдалении от больших потоков частиц, может быть построена сильно упрощённой без потери реалистичности моделирования.
%Так, например, резьба болта, находящегося близко к области, где проходит пучок, может оказать влияние на функционирование всей установки, а та же резьба где-то за пределами геометрического аксептанса не даст ровно никого эффекта может быть упрощена до цилиндра. Более того, без ущерба реалистичности моделирования упрощения могут носить неожиданно масштабный характер. Например, где-то рассматриваемый массив болтов может быть вообще проигнорирован, а пространство в отверстиях заполнено материалом корпуса.

Таким образом с целью упрощения взаимодействия физиков и инженеров было принятно решение не пытаться разработать конвертеры или какие-либо новые универсальные способы представления геометрии, а сосредоточиться на облегчении существующей процедуры за счёт плавной корректировки привычных методов и предоставления новых инструментов как физикам, так и инженерам. ``CATIA-GDML geometry builder'' --- это как раз набор таких инструментов. Он описан в~\ref{sec:secBuilder} вместе с предлагаемой организацией рабочего процесса и реальным случаем использования для проектирования детектора RICH эксперимента CBM.
