\section{Инструменты передачи геометрии}\label{sec:ExistSols}

\subsection{FASTRAD}\label{sec:secFastrad}

% \todo

FASTRAD --- это программный продукт фирмы TRAD для выполнения анализа радиационных доз~\cite{FASTRAD}.
FASTRAD включает в себя подсистемы геометрического моделирования, расчёта, постпроцессинга и визуализации.
В FASTRAD имеется как собственная реализация метода Монте-Карло для моделирования прохождения частиц через вещество, так и возможность экспортировать постановку задачи для расчёта с помощью GEANT4.
Имеется возможность интеграции с САПР с помощью форматов STEP и IGES. При этом, однако, импортированное из САПР BREP-описание геометрии триангулируется и дальнейшее моделирование выполняется с фасеточной геометрией.

% \subsection{SimpleGeo}\label{sec:secSimpleGeo}
% \cite{SimpleGeo}
% \todo

\subsection{CADtoROOT interface}\label{sec:secCinzia}

Данный интерфейс~\cite{Cinzia} разрабатывается в CERN, является частью ROOT и основан на Open CASCADE Technology~\cite{OCCtech} --- открытой платформе для разработки 3d-приложений. Он позволяет выполнить анализ CSG-описания установки в менеджере геометрии ROOT и экспортировать его BREP-описание в STEP файл.

\subsection{CAD converter}\label{sec:secTobias}

Программный продукт ``CAD converter''~\cite{Tobias} разрабатыватся в коллаборации PANDA. ``CAD converter'' импортирует входной файл формата STEP или IGES средствами OpenCASCADE, выполняет анализ BREP-форм и пытается распознать в них примитивы. После этого для найденных примитивов определяются параметры и матрица позиционирования, выполняется их экспорт в ROOT-файл, а нераспознанные формы экспортируются в другой STEP файл.
