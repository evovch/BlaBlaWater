% ================================================================================================================
%  ____        _ _     _                                             _           
% | __ ) _   _(_) | __| | ___ _ __     ___  ___ ___ _ __   __ _ _ __(_) ___  ___ 
% |  _ \| | | | | |/ _` |/ _ \ '__|   / __|/ __/ _ \ '_ \ / _` | '__| |/ _ \/ __|
% | |_) | |_| | | | (_| |  __/ |      \__ \ (_|  __/ | | | (_| | |  | | (_) \__ \
% |____/ \__,_|_|_|\__,_|\___|_|      |___/\___\___|_| |_|\__,_|_|  |_|\___/|___/
%                                                                                
% ================================================================================================================

\section[Основные сценарии работы с ``CATIA-GDML geometry builder'']{Основные сценарии работы с\\ ``CATIA-GDML geometry builder''}\label{sec:secScenarios}

% ================================================================================================================
%  __  __  ____                                       __                                            _       _     
% |  \/  |/ ___|     __ _  ___  ___  _ __ ___        / _|_ __ ___  _ __ ___      ___  ___ _ __ __ _| |_ ___| |__  
% | |\/| | |        / _` |/ _ \/ _ \| '_ ` _ \      | |_| '__/ _ \| '_ ` _ \    / __|/ __| '__/ _` | __/ __| '_ \ 
% | |  | | |___    | (_| |  __/ (_) | | | | | |_    |  _| | | (_) | | | | | |   \__ \ (__| | | (_| | || (__| | | |
% |_|  |_|\____|    \__, |\___|\___/|_| |_| |_(_)   |_| |_|  \___/|_| |_| |_|   |___/\___|_|  \__,_|\__\___|_| |_|
%                   |___/                                                                                         
% ================================================================================================================

\subsection{Создание MC-геометрии с нуля}\label{sec:secMCfromScratch}

В процессе разработки ``Builder'' был выработан стандартный алгоритм создания геометрии. Первый шаг --- создание нового документа типа CATProduct и его сохранение на диск. В дальнейшем это будет единственный документ типа CATProduct и он будет представлять собой модель всей экспериментальной установки. Второй этап --- наполнение продукта описанием объёмов без описания взаимосвязей между ними. Для этого используется макрос \macroname{AddNewPart}, который автоматически открывает в отдельном окне новый документ типа CATPart, сформированный из специального шаблона и соответствующий создаваемому объёму. Система переходит в режим редактирования детали, где доступны только два макроса \macroname{AddShape} и \macroname{Poly} для создания формы объёма. Здесь же можно и задать имя материала объёма. По окончании редактирования нового объёма в отдельном окне пользователь должен сохранить активный документ и закрыть это окно. CATIA при этом возвращается к редактированию продукта. После того, как созданы объёмы, заданы формы и, возможно, имена материалов, алгоритм подразумевает задание иерархии объёмов, то есть вставку и позиционирование одних объёмов внутри других. Для этого в ``Builder'' существует целый ряд макропрограмм для создания различных типов взаимосвязей --- \macroname{Inserter}, \macroname{ArrayMaker}, \macroname{Replica}. После того как выполнено размещение дочернего объёма $A$ в материнском объёме $B$, пользователь может указать поворот и сдвиг, задающие матрицу позиционирования $A$ в $B$. Для упрощения расчётов в некоторых случаях очень удобно применять макропрограммы \macroname{PointToPointAligner} (\macroname{Pt2PtAligner}), \macroname{Mover} и \macroname{Measure}. Для удобного редактирования материалов всех объёмов был разработан менеджер материалов \macroname{MaterialsManager}, который обычно имеет смысл вызывать перед экспортом для проверки ранее заданных имён материалов, либо назначения новых. Также перед экспортом рекомендуется проверить модель на наличие ошибок с помощью макроса \macroname{Checker}. В конце выполняется экспорт во внешний GDML файл макросом \macroname{CATIA2GDML}. Отдельно стоят макропрограммы \macroname{Duplicator} для создания множественных идентичных, но не связанных, параметризованных подсборок и обратный конвертер \macroname{GDML2CATIA} для импорта GDML файла.

% ================================================================================================================
%  __  __  ____      __                          ____    _    ____  
% |  \/  |/ ___|    / _|_ __ ___  _ __ ___      / ___|  / \  |  _ \ 
% | |\/| | |       | |_| '__/ _ \| '_ ` _ \    | |     / _ \ | | | |
% | |  | | |___    |  _| | | (_) | | | | | |   | |___ / ___ \| |_| |
% |_|  |_|\____|   |_| |_|  \___/|_| |_| |_|    \____/_/   \_\____/ 
%                                                                   
% ================================================================================================================

\subsection[Создание MC-геометрии на основе существующей CAD-геометрии]{Создание MC-геометрии на основе существующей\\ CAD-геометрии}\label{sec:secMCfromCAD}

Инженерная модель может быть создана как в CATIA, так и с помощью любой другой САПР и затем передана в CATIA, например, с помощью файла STEP~(см. секцию~\ref{sec:secGeoFormats}).
% Отсюда вырезался кусок про STEP
В результате импорта получается сборка в документе типа CATProduct, которая содержит в себе одну или несколько деталей в качестве компонентов.
Как было сказано выше в начале секции, в соответствии со структурой ``CATIA-GDML geometry builder'' вся геометрия установки создаётся в одном документе типа CATProduct. 
Пользователь имеет возможность либо открыть оба документа типа CATProduct параллельно, в одной или нескольких сессиях CATIA, либо поместить один продукт в другой в качестве компонента. Чтобы пользоваться макропрограммами ``Builder'', необходимо чтобы продукт с MC-сборкой находился на верхнем уровне, поэтому имеет смысл помещать импортированную инженерную модель в продукт с MC-моделью. В этом случае предоставляется возможность в одном геометрическом пространстве работать с исходной САПР-моделью и разрабатываемой MC-моделью. Так, например, удобно подстраивать размер и позиционирование создаваемых объёмов, визуально сравнивая с соответствующими элементами САПР-модели. Нередко для этого необходимо совместить начала координат двух моделей перемещением и, может быть, поворотом САПР-модели, т.к. начало координат MC-модели должно совпадать с началом координат продукта верхнего уровня. Это можно сделать средствами геометрических ограничений в режиме создания сборок (Assembly design), а иногда и средствами преобразования координат в режиме создания деталей (Part Design). Стоит, однако, учитывать, что перед экспортом MC-модели в GDML необходимо удалить компонент с САПР-моделью, т.к. конвертер \macroname{CATIA2GDML} не сможет определить, что этот компонент не является частью описания MC-модели и должен быть проигнорирован.
% \todo
% \textbf{Возможно, показать картинку - скриншот дерева, где в МС-сборку вставлен продукт из степа}
При таком режиме работы удобно пользоваться макропрограммами \macroname{PointToPointAligner} (см. секцию~\ref{sec:secMacroPtPAligner}) и \macroname{Measure} (см. секцию~\ref{sec:secMacroMeasure}).

% ================================================================================================================
%  __  __  ____     _             ____    _    ____  
% |  \/  |/ ___|   | |_ ___      / ___|  / \  |  _ \ 
% | |\/| | |       | __/ _ \    | |     / _ \ | | | |
% | |  | | |___    | || (_) |   | |___ / ___ \| |_| |
% |_|  |_|\____|    \__\___/     \____/_/   \_\____/ 
%                                                    
% ================================================================================================================

\subsection[Создание MC-геометрии на основе существующей MC-геометрии]{Создание MC-геометрии на основе существующей\\ MC-геометрии}\label{sec:secMCtoCAD}

``Обратный'' конвертер \macroname{GDML2CATIA} позволяет импортировать в CATIA MC-модель, записанную в GDML файл.

% \todo hyphenation trick
GDML имеет открытую спецификацию, его поддержка реализована в GEANT4 и ROOT. Соответственно, если имеется модель в GEANT4 или ROOT, а также любом ROOT"=наследованном пакете, включая FairRoot-наследованные (как, например, CbmRoot), то её можно стандартным средствами экспортировать в GDML без потери информации. Для этого, однако, функциональность GDML должна быть активирована соответствующим флагом при компиляции.

Последовательность работы простая --- создаётся новый документ типа CATProduct и запускается макропрограмма \macroname{GDML2CATIA}, где выбирается входной GDML файл и запускается импорт. Данную процедуру рекомендуется выполнять в новой сессии CATIA, т.е. после её перезапуска, т.к. замечено, что CATIA иногда хранит в фоновом режиме информацию о ранее открытых документах, а это может приводить к коллизии имён и, следовательно, некорректному результату. По окончании импорта работа над моделью установки выполняется без каких-либо особенностей. Если требуется создать другую модель, опираясь как-то на импортированную, то, также как и в случае работы с инженерной моделью, можно либо открыть два продукта параллельно, либо поместить импортированную модель внутрь разрабатываемой в качестве дочернего компонента продукта вернего уровня.
