% ================================================================================================================
%  __  __           _      _                     _       _   _             
% |  \/  | ___   __| | ___| |     _____   _____ | |_   _| |_(_) ___  _ __  
% | |\/| |/ _ \ / _` |/ _ \ |    / _ \ \ / / _ \| | | | | __| |/ _ \| '_ \ 
% | |  | | (_) | (_| |  __/ |   |  __/\ V / (_) | | |_| | |_| | (_) | | | |
% |_|  |_|\___/ \__,_|\___|_|    \___| \_/ \___/|_|\__,_|\__|_|\___/|_| |_|
%                                                                          
% ================================================================================================================

\subsection{История развития MC-модели CBM RICH}\label{sec:secRICHgeoHistory}

На протяжении нескольких лет работы над проектированием детектора было создано несколько версий MC-геометрии CBM RICH. В основном, каждая следующая версия либо уточняла предыдущую, либо включала в себя обновления каких-то подсистем. Стоит отметить несколько устаревших на данный момент версий, в которых были внесены значительные изменения в соответствии с обновляющимся проектом детектора.

% ================================================================================================================
\subsubsection{Модель с примитивным фотодетектором}\label{sec:secPrimitiveCamera}

Первоначально рассматривался вариант, в котором фоточувствительная камера была составлена из четырёх плоскостей, расположенных симметрично относительно горизонтальной и вертикальной плоскостей. Начиная с самых ранних версий в модели RICH в CbmRoot камера была выполнена с помощью тонких пластин из активного материала, причём размер этих пластин был выбран так, чтобы полностью покрывать аксептанс, никак не соотносясь с реальными возможностями. На \figref{fig:PrimitivePhotodetector} показана часть модели CBM RICH с примитивным фотодетектором.

\begin{figure}[H]
\centering
\includegraphics[width=0.6\textwidth]{pictures/PrimitivePhotodetector.png}
\caption{Часть одной из наиболее ранних MC-моделей CBM RICH, в которой фотодетектор был представлен тонкими чувствительными боксами.}
\label{fig:PrimitivePhotodetector}
\end{figure}

% ================================================================================================================
\subsubsection{Модель без магнитного экрана}\label{sec:secNoMagScreen}

Длительное время MC-модель CBM RICH не имела магнитного экрана, см. \figref{fig:MCgeoMirrorsEvolution}(слева). По этой причине не было необходимости создавать дополнительное пространство для выступающей части, что делало форму материнского объёма значительно более простой.

% ================================================================================================================
\subsubsection{Модель с промежуточным объёмом для магнитного экрана}\label{sec:secInterVolMagScreen}

В первой версии MC-модели с магнитным экраном был введён дополнительный промежуточный объём, позиционированный параллельно системе координат объёма радиатора. В него были на одном уровне помещены пластины, представляющие стенки экрана и ещё два контейнера --- с МА~ФЭУ и электроникой (см. \figref{fig:ShieldingBoxMC}). На момент написания работы магнитный экран выполнен как набор пластин, помещённых непосредственно в контейнер для камеры. Принципиальным отличием является то, что поворот и позиционирование экрана теперь выполняется вместе со всей камерой в системе координат объёма радиатора, в то время как в старой модели --- отдельно в системе координат промежуточного контейнера.

% ================================================================================================================
\subsubsection{Модель с зазором между зеркалами}\label{sec:secMirrorsEvolution}

Значительным улучшением в проекте CBM RICH стал переход от формы зеркал, симметричной относительно горизонтальной плоскости, к особой форме, позволяющей стыковать два зеркала практически без зазора (см. \figref{fig:MCgeoMirrorsEvolution}). Изначально рассматривался вариант, в котором одно зеркало выполнено из долей двух типов. Тогда для того, чтобы центр сферической поверхности располагался на расстоянии (над для верхнего зеркала и под --- для нижнего), необходимо было поворачивать каждое зеркало. Новые зеркала не требует поворота, т.к. они имеют форму соответствующего сегмента сферы. Однако это приводит к необходимости иметь не 2, а 4 типоразмера сегментов зеркал и немного усложняет их изготовление (см. \figref{fig:Mirrors4types}).

\begin{figure}[H]
\begin{minipage}[b]{0.495\textwidth}
\includegraphics[width=1.0\textwidth]{pictures/RICH_MC_evolution_before.png}
\end{minipage}
\hspace{0.01\textwidth}
\begin{minipage}[b]{0.495\textwidth}
\includegraphics[width=1.0\textwidth]{pictures/RICH_MC_evolution_after.png}
\end{minipage}
\caption{Модель со старыми зеркалами (слева) и модель с новыми зеркалами (справа).}
\label{fig:MCgeoMirrorsEvolution}
\end{figure}

\begin{figure}[H]
\centering
\includegraphics[width=0.3\textwidth]{pictures/Mirrors_4types_of_segments.png}
\caption{4 типоразмера сегментов зеркал, позволяющие собрать фокусирующую систему без зазоров.}
\label{fig:Mirrors4types}
\end{figure}

% ================================================================================================================
\subsubsection{Модель с ``малой рамой''}\label{sec:secModelWithSmallFrame}

На раннем этапе была предложена конструкция опор зеркал, которая в дальнейшем была отвергнута из-за слишком большого количества материала в аксептансе. На \figref{fig:SmallFrameCADandMC} показана модель опор зеркал в САПР (слева) и в CbmRoot (справа).

% \todo
% написать что-то о реализации
% Много трапов, поэтому был разработан hollow trap template и trap-creator, который потом эволюционировал в Primitive_creator

\begin{figure}[H]
\begin{minipage}[b]{0.495\textwidth}
\includegraphics[width=1.0\textwidth]{pictures/Frame_small_CAD.png}
\end{minipage}
\hspace{0.01\textwidth}
\begin{minipage}[b]{0.495\textwidth}
\includegraphics[width=1.0\textwidth]{pictures/Frame_small_MC2.png}
\end{minipage}
\caption{Ранняя модель опор зеркал в САПР CATIA~V5 (слева) и в CbmRoot (справа).}
\label{fig:SmallFrameCADandMC}
\end{figure}

% ================================================================================================================
\subsubsection{Модель с репликой долей зеркал}\label{sec:secModelWithReplMirrors}

% \todo
ОБратим внимание, что
Где какие оси, что вокруг чего крутится, реплицируется, ось одна перпенд. другой и т.д.
Края кусочков - конусы и плоскости.

\begin{figure}[H]
\begin{minipage}[t]{0.495\textwidth}
\includegraphics[width=0.95\textwidth]{pictures/CbmRichMirror1.png}
\end{minipage}
\hspace{0.01\textwidth}
\begin{minipage}[t]{0.495\textwidth}
\includegraphics[width=0.95\textwidth]{pictures/CbmRichMirror2.png}
\end{minipage}
\caption{}
\label{fig:ReplicaMirror}
\end{figure}
