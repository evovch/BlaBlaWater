Процесс проектирования современной экспериментальной установки подразумевает разнообразное компьютерное моделирование этой установки.

%В первую очередь выполняется компьютерное геометрическое моделирование в трёхмерном пространстве с целью получения конструкторской документации и анализа расположения элементов в пространстве. Геометрическая модель для этих целей обычно строится средствами систем автоматизированного проектирования (САПР), в которых стандартным способом представления геометрической информации является граничное представление (BREP).
%\textbf{Далее представление геометрической информации в САПР мы будем называть просто BREP, хотя строго говоря это не совсем верно.}
%Также, как и в любой другой прикладной области, необходимо выполнять многочисленные расчёты, которые нередко требуют геометрическую модель в качестве входных данных. Так, например, в инженерно-конструкторской среде широкое распространение получил метод конечных элементов (МКЭ) для решения задач прочности и устойчивости механических конструкций, динамики жидкостей и газов и т.д.
%Применяются соответствующие способы описания геометрии
%ссылка на соответствующую секцию

Отличительной особенностью сферы физики частиц является то, что в процессе проектирования установки помимо типовых расчётов требуется выполнение моделирования прохождения частиц через материал, которое чаще всего выполняется физиками, формулирующими требования к конструкции установки, но в общей массе не владеющими САПР. Такое моделирование достаточно специфично, но оно также выполняется над геометрической моделью, в идеальной ситуации --- максимально подробной, совпадающей с полной детальной моделью, полученной инженерами-конструкторами с помощью САПР. Также стоит отметить, что процесс конструирования, в том числе получения инженерной геометрической модели, и процесс моделирования физики не имеют чётко определённого порядка и тесно между собой переплетены. В результате обоих процессов уточняются геометрические параметры деталей, компоновка узлов, применяемые материалы, и т.д. Это приводит к необходимости постоянного обмена геометрической информацией.

Как было сказано выше, инженеры для получения геометрической модели используют САПР. Во многих физических лабораториях, включая CERN, GSI и ОИЯИ, применяется САПР CATIA~v5. Моделирование взаимодействия частиц с материалом широко применяет метод Монте-Карло (MC) и реализовано в соответствующих программных пакетах, многие из которых основаны на фреймворках GEANT4 или GEANT3, разработанных в CERN. Также часто применяют поход Virtual Monte-Carlo (VMC), в котором все процедуры, связанные с геометрией, поручены системе ROOT. Все перечисленные физические пакеты (GEANT3, GEANT4, ROOT, далее коротко GEANT/ROOT) используют представление геометрии, принципиально отличающееся от BREP. Модели для GEANT/ROOT часто называют MC-моделями. Это отличие состоит из двух пунктов, подробно описанных в~\ref{sec:secROOTvsCAD}, и приводит к невозможности прямого обмена геометрическими моделями между физиками и инженерами. Существует теоретическая возможность прямой конвертации из представления, принятого в GEANT/ROOT, в BREP, однако в процессе работы не было найдено существующей реализации подобного перехода. Конвертация в обратном направлении до настоящего времени не была математически описана, хотя теоретически также представляется возможной.

Алгоритмы проведения частиц, реализованные в GEANT/ROOT, оптимизированы для соответствующего описания геометрии, применяемого в этих пакетах. Подходы геометрического моделирования, принятые в САПР, обеспечивают максимально эффективную работу как ЭВМ, так и инженеров, в частности за счёт того, что эти подходы интуитивно понятны человеку. Главным фактором против прямой конвертации в том или ином направлении является то, что она имеет малую практическую пользу. Одна и та же геометрическая модель с точки зрения разных задач может быть одновременно оптимальна и, наоборот, избыточна или недостаточна. Это просто понять на следующем примере. С точки зрения инженерного проекта массив болтов, вкрученных в корпус, представляет собой важную информацию. В чертежах и другой конструкторской документации ошибка в точном положении отверстий, их диаметре, типе резьбы болтов и т.д. может привести к невозможности собрать продукт после изготовления отдельных компонентов. В то же время, в САПР принято не хранить, и следовательно не визуализировать, витки резьбы с целью снижения нагрузки на графический адаптер ЭВМ. Это значит, что резьба присутствует только формально, в документации, а геометрическая модель имеет лишь условное обозначение резьбы в соответствующем месте. С точки зрения моделирования прохождения частиц через материал в зависимости от расположения в общей установке подобные подробности могут оказаться как критическими, так и наоборот излишними и вызывающими значительное увеличение времени выполнения моделирования. Так, например, резьба болта, находящегося близко к области, где проходит пучок, может оказать влияние на функционирование всей установки, а та же резьба где-то за пределами геометрического аксептанса не даст ровно никого эффекта может быть упрощена до цилиндра. Более того, без ущерба реалистичности моделирования упрощения могут носить неожиданно масштабный характер. Например, где-то рассматриваемый массив болтов может быть вообще проигнорирован, а пространство в отверстиях заполнено материалом корпуса.

В связи с этим в GEANT/ROOT принято иметь несколько моделей одной и той же установки, имеющих разный уровень подробностей. Чем выше уровень подробностей --- тем больше времени занимает выполнение моделирования. Для оценочных расчётов удобно применять грубые модели, для точного определения каких-либо характеристик --- подробные модели. В САПР же подобная проблема решается другим образом. Так, например, в САПР CATIA~v5 присутствует возможность автоматического огрубления геометрической модели для снижения нагрузки на графический адаптер и повышения частоты кадров при динамической визуализации трёхмерных объектов. Это становится актуально, когда количество треугольников, которые необходимо визуализировать, составляет десятки миллионов.

% Упомянуть FastMC

Принимая во внимание развитие вычислительной техники, в особенности резкое повышение производительности графических карт, их доступность широким массам, и вообще увеличение их значимости в вычислениях общего назначения, представляется возможным разработка новых алгоритмов проведения частиц через материал, учитывающих особенности геометрического представления в САПР. Более того, возможна также некоторая корректировка подходов САПР к геометрическому моделированию с целью повышения совместимости с пакетами проведения частиц.
\textbf{Один из возможных подходов --- реализация полигонального (см. секцию~\ref{sec:secGeoPoly}) твердотельного моделирования со строгим ограничением замкнутости оболочек, представляющих границы тела.}
Однако следует учитывать следующие факты, мешающие движению в данном направлении. Во-первых, САПР --- это в большинстве своём коммерческое программное обеспечение с закрытым исходным кодом, а геометрическое ядро САПР --- базовая составляющая, которую отлаживают десятилетиями. Внесение изменений в столь важную компоненту коммерческого продукта, вероятно, будет проблемным даже при наличии интереса со стороны фирмы-разработчика. Во-вторых, в обеих сферах накоплен огромный массив моделей, применяемых для поддержки изделий на всех этапах жизненного цикла, даже после окончания процесса проектирования. GEANT/ROOT модели могут применяться для выполнения моделирования даже после того, как физическая экспериментальная установка уже собрана, а иногда даже и уже разобрана.
\todo

Таким образом с целью упрощения взаимодействия физиков и инженеров было принятно решение не пытаться разработать конвертеры или какие-либо новые универсальные способы представления геометрии, а сосредоточиться на облегчении существующей процедуры за счёт плавной корректировки привычных методов и предоставления новых инструментов как физикам, так и инженерам. ``CATIA-GDML geometry builder'' --- это как раз набор таких инструментов. Он описан в~\ref{sec:secBuilder} вместе с предлагаемой организацией рабочего процесса и реальным случаем использования для проектирования детектора RICH эксперимента CBM.
