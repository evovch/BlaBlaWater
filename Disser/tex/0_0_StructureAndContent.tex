\section{Структура и содержание}\label{sec:secStructureAndContent}

Диссертация состоит из настоящего введения, пяти глав и заключения.

В первой главе
описываются условия эксплуатации, компоновка и основные свойства детекторов эксперимента CBM;
обсуждается важность точного описания и оптимизации конструкции детекторов в свете жёстких условий эксплуатации;
формулируется задача разработки инструментария для обмена геометрической информацией между САПР и средами Монте-Карло моделирования прохождения частиц через вещество (Geant4/ROOT) и облегчения создания детальных геометрических моделей для Geant4/ROOT;
детально описывается конструкция детектора CBM RICH и проводится сравнение с аналогичными приборами из ряда других экспериментов;
формулируются конкретные задачи, связанные с описанием геометрии детектора CBM RICH в среде Монте-Карло;
обсуждается концепция системы считывания и сбора данных эксперимента CBM и воплощение этой концепции в системе считывания и сбора данных детектора CBM RICH;
формулируется задача на исследование прототипа системы считывания и сбора данных указанного детектора.

Во второй главе
обсуждаются некоторые наиболее распространённые способы представления геометрических моделей в ЭВМ, используемые в различном ПО для решения различных вычислительных задач;
рассматриваются предпосылки и принципы создания инструментария, так называемого ``CATIA-GDML geometry builder'', для обмена геометрической информацией между САПР и средами Монте Карло моделирования прохождения частиц через вещество (Geant4/ROOT) и облегчения создания детальных геометрических моделей для Geant4/ROOT;
обсуждаются реализация отображения примитивов и иерархии объемов Geant4/ROOT на дерево построений в среде CATIA и набор созданных макропрограмм, входящих в ``CATIA-GDML geometry builder'';
описывается методика применения ``CATIA-GDML geometry builder'' и приводятся некоторые примеры.

\textbf{В третьей главе
приводится описание геометрической модели CBM RICH в Geant4/ROOT-совместимом формате, созданной с помощью ``CATIA-GDML geometry builder''. Подробно рассмотрены особенности модели, связанные с конкретными задачами по разработке и оптимизации детектора --- применение параметризации для моделирования отклонения индивидуальных зеркал фокусирующей системы, использование стандартных средств CATIA для максимально точного моделирования количества пассивного материала в аксептансе, ...
Третья глава посвящена применению ``CATIA-GDML geometry builder'' для детектора CBM RICH. В ней подробно рассмотрены задачи описания геометрии механических конструкций детектора, оптимизации? системы крепления и позиционирования зеркал, учета эффектов, связанных с отклонением позиционирования зеркал от номинального, размещения и экранирования от магнитного поля фотодетекторов. Также описаны созданные геометрические модели и проведенные с их помощью исследования (по материалам коллег) свойств детектора CBM RICH.
}

В четвёртой главе описаны архитектура бестриггерной системы считывания и сбора данных CBM RICH, разработанные модули ПО, необходимые для сбора и анализа данных, а также экспериментальные установки, позволившие осуществить всестороннее исследование прототипа указанной системы.

Пятая глава посвящена анализу данных пучковых и лабораторных тестов прототипа детектора CBM RICH и результатам исследования свойств и характеристик прототипа системы считывания и сбора данных. Здесь же, на основании проведенных исследований, даются рекомендации по модификации следующей версии прототипа системы считывания и сбора данных.

В заключении приводятся основные результаты работы и выражаются благодарности.
