\textbf{На защиту выносятся следующие результаты:}

\begin{enumerate}
\item Разработка методологии и реализация ``CATIA-GDML geometry builder'', средства построения сложной, основанной на инженерном дизайне геометрии детекторов для моделирования прохождения и взаимодействия частиц.
\item Применение ``CATIA-GDML geometry builder'' для построения беспрецедентно точного параметризованного описания геометрии CBM~RICH в среде CbmRoot.
\item Реализация прототипа системы считывания и сбора данных CBM~RICH и проведение его тестов на пучке в составе полнофункционального прототипа этого детектора а также дополнительных тестов на лабораторном стенде.
\item Разработка алгоритмов и программного обеспечения для приема, упаковки и передачи бестриггерного потока данных, для калибровки точного времени и относительных задержек каналов и для построения событий из потока данных с детектора CBM RICH в среде CbmRoot.
\item Результаты комплексного исследования временных свойств канала считывания и сбора данных для CBM~RICH, реализованного на основе многоанодного ФЭУ H12700 с системой динодов ``metal channel'', специально разработанных передней электроники типа предусилитель-дискриминатор и высокоточного ВЦП с последующим прямым вводом данных в единую среду моделирования, сбора и анализа данных CbmRoot.
\item Исследование временных свойств нанесенного на окно МА ФЭУ сместителя спектра при возбуждении черенковскими фотонами.
\item Сравнительный анализ особенностей считывания многоанодного ФЭУ временным и аналоговым трактами.
\item Исследование характеристик детектора CBM~RICH с учетом неидеальности геометрии и шумов электроники.
\end{enumerate}
