\chapter{Геометрия}\label{sec:chapGeometry}

Процесс проектирования современной экспериментальной установки подразумевает разнообразное компьютерное моделирование этой установки. В первую очередь выполняется компьютерное геометрическое моделирование в трёхмерном пространстве с целью получения конструкторской документации и анализа расположения элементов в пространстве. Геометрическая модель для этих целей обычно строится средствами систем автоматизированного проектирования (САПР), в которых стандартным способом представления геометрической информации является граничное представление (BREP).

\textbf{Далее представление геометрической информации в САПР мы будем называть просто BREP, хотя строго говоря это не совсем верно.}

Также, как и в любой другой прикладной области, необходимо выполнять многочисленные расчёты, которые нередко требуют геометрическую модель в качестве входных данных. Так, например, в инженерно-конструкторской среде широкое распространение получил метод конечных элементов (МКЭ) для решения задач прочности и устойчивости механических конструкций, динамики жидкостей и газов и т.д. МКЭ получил своё название от способа разбиения расчётной области на элементарные блоки --- конечные элементы. В простом случае расчётной областью является пространство заполненное материалом, т.е. сама деталь, а конечным элементом --- тетраэдр. Такую модель, в которой деталь представлена множеством конечных элементов, называют КЭ-моделью. Существуют алгоритмы, позволяющие эффективно получить разбиение исходной геометрической модели, например представленной с помощью BREP и построенной в САПР, на конечные элементы. Многие алгоритмы основаны на триангуляции Делоне, разработанной в начале 20~века.

Отличительной особенностью сферы физики частиц является то, что в процессе проектирования установки помимо типовых расчётов требуется выполнение моделирования прохождения частиц через материал, которое чаще всего выполняется физиками, формулирующими требования к конструкции установки, но в общей массе не владеющими САПР. Такое моделирование достаточно специфично, но оно также выполняется над геометрической моделью, в идеальной ситуации --- максимально подробной, совпадающей с полной детальной моделью, полученной инженерами-конструкторами с помощью САПР. Также стоит отметить, что процесс конструирования, в том числе получения инженерной геометрической модели, и процесс моделирования физики не имеют чётко определённого порядка и тесно между собой переплетены. В результате обоих процессов уточняются геометрические параметры деталей, компоновка узлов, применяемые материалы, и т.д. Это приводит к необходимости постоянного обмена геометрической информацией.

Как было сказано выше, инженеры для получения геометрической модели используют САПР. Во многих физических лабораториях, включая CERN, GSI и ОИЯИ, применяется САПР CATIA~v5. Моделирование взаимодействия частиц с материалом широко применяет метод Монте-Карло (MC) и реализовано в соответствующих программных пакетах, многие из которых основаны на фреймворках GEANT4 или GEANT3, разработанных в CERN. Также часто применяют поход Virtual Monte-Carlo (VMC), в котором все процедуры, связанные с геометрией, поручены системе ROOT. Все перечисленные физические пакеты (GEANT3, GEANT4, ROOT, далее коротко GEANT/ROOT) используют представление геометрии, принципиально отличающееся от BREP. Модели для GEANT/ROOT часто называют MC-моделями. Это отличие состоит из двух пунктов, подробно описанных в~\ref{sec:ROOTvsCAD}, и приводит к невозможности прямого обмена геометрическими моделями между физиками и инженерами. Существует теоретическая возможность прямой конвертации из представления, принятого в GEANT/ROOT, в BREP, однако в процессе работы не было найдено существующей реализации подобного перехода. Конвертация в обратном направлении до настоящего времени не была математически описана, хотя теоретически также представляется возможной.

Алгоритмы проведения частиц, реализованные в GEANT/ROOT, оптимизированы для соответствующего описания геометрии, применяемого в этих пакетах. Подходы геометрического моделирования, принятые в САПР, обеспечивают максимально эффективную работу как ЭВМ, так и инженеров, в частности за счёт того, что эти подходы интуитивно понятны человеку. Главным фактором против прямой конвертации в том или ином направлении являтся то, что она имеет малую практическую пользу. Одна и та же геометрическая модель с точки зрения разных задач может быть одновременно оптимальна и, наоборот, избыточна или недостаточна. Это просто понять на следующем примере. С точки зрения инженерного проекта массив болтов, вкрученных в корпус, представляет собой важную информацию. В чертежах и другой конструкторской документации ошибка в точном положении отверстий, их диаметре, типе резьбы болтов и т.д. может привести к невозможности собрать продукт после изготовления отдельных компонентов. В то же время, в САПР принято не хранить, и следовательно не визуализировать, витки резьбы с целью снижения нагрузки на графический адаптер ЭВМ. Это значит, что резьба присутствует только формально, в документации, а геометрическая модель имеет лишь условное обозначение резьбы в соответствующем месте. С точки зрения моделирования прохождения частиц через материал в зависимости от расположения в общей установке подобные подробности могут оказаться как критическими, так и наоборот излишними и вызывающими значительное увеличение времени выполнения моделирования. Так, например, резьба болта, находящегося близко к области, где проходит пучок, может оказать влияние на функционирование всей установки, а та же резьба где-то за пределами геометрического аксептанса не даст ровно никого эффекта может быть упрощена до цилиндра. Более того, без ущерба реалистичности моделирования упрощения могут носить неожиданно масштабный характер. Например, где-то рассматриваемый массив болтов может быть вообще проигнорирован, а пространство в отверстиях заполнено материалом корпуса.

В связи с этим в GEANT/ROOT принятно иметь несколько моделей одной и той же установки, имеющих разный уровень подробностей. Чем выше уровень подробностей --- тем больше времени занимает выполнение моделирования. Для оценочных расчётов удобно применять грубые модели, для точного определения каких-либо характеристик --- подробные модели. В САПР же подобная проблема решается другим образом. Так, например, в САПР CATIA~v5 присутствует возможность автоматического огрубления геометрической модели для снижения нагрузки на графический адаптер и повышения частоты кадров при динамической визуализации трёхмерных объектов. Это становится актуально, когда количество треугольников, которые необходимо визуализировать, составляет десятки миллионов.

Принимая во внимание развитие вычислительной техники, в особенности резкое повышение производительности графических карт, их доступность широким массам, и вообще увеличение их значимости в вычислениях общего назначения, представляется возможным разработка новых алгоритмов проведения частиц через материал, учитывающих особенности геометрического представления в САПР. Более того, возможна также некоторая корректировка подходов САПР к геометрическому моделированию с целью повышения совместимости с пакетами проведения частиц. Однако следует учитывать следующие факты, мешающие движению в данном направлении. Во-первых, САПР --- это в большинстве своём коммерческое программное обеспечение с закрытым исходным кодом, а геометрическое ядро САПР --- базовая составляющая, которую отлаживают десятилетиями. Внесение изменений в столь важную компоненту коммерческого продукта, вероятно, будет проблемным даже при наличии интереса со стороны фирмы-разработчика. Во-вторых, в обоих сферах накоплен огромный массив моделей, применяемых для поддержки изделий на всех этапах жизненного цикла, даже после окончания процесса проектирования. GEANT/ROOT модели могут применяться для выполнения моделирования даже после того, как физическая экспериментальная установка уже собрана.

Таким образом с целью упрощения взаимодействия физиков и инженеров было принятно решение не пытаться разработать конвертеры или какие-либо новые универсальные способы представления геометрии, а сосредоточиться на облегчении существующей процедуры за счёт плавной корректировки привычных методов и предоставления новых инструментов как физикам, так и инженерам. ``CATIA-GDML geometry builder'' --- это как раз набор таких инструментов. Он описан в~\ref{sec:Builder} вместе с предлагаемой организацией рабочего процесса и реальным случаем использования для проектирования детектора RICH эксперимента CBM.

\section{Сравнение представления геометрии в GEANT/ROOT и САПР}\label{sec:ROOTvsCAD}

Разница между двумя способами описания геометрической информации в САПР и пакетах моделирования прохождения частиц через материал GEANT/ROOT заключается в двух пунктах. Во-первых, отличается способ задания геометрических форм. В САПР применяется граничное представление (BREP), для описания которого используются понятия типа <<поверхность>>, <<грань>>, <<ребро>>, <<кривая>>, и за которыми стоят соответствующие уравнения, описывающие эти объекты в пространстве. В GEANT/ROOT применяется конструктивная твердотельная геометрия (CSG), которая оперирует понятиями <<примитив>> и <<Булева операция>>. Очевидно, что и за этими объектами также стоят конкретные уравнения, описывающие кривые и поверхности, однако есть существенное различие описанное ниже. Во-вторых, отличается способ задания взаимоотношения форм в пространстве. В САПР, по аналогии с тем, как человек воспринимает окружающий мир, присутствует некоторое бесконечное окружающее пространство без материала, а все предметы находятся в этом пространстве. Невозможна такая ситуация, чтобы один объект находился внутри другого --- в таком случае подразумевается, что во втором есть соответствующая полость, освобождающая место под первый объект. В GEANT/ROOT для описания взаимоотношения форм используется иерархия объёмов. Это объясняется тем, что такой метод более удобен для описания геометрии, где главной задачей является однозначное задание материала в каждой точке пространства. Вводится понятие объёма --- сущности, имеющей форму и материал. Из всех объёмов выбирается один, называемый объёмом верхнего уровня, а остальные помещены либо в него, либо в какой-то другой, формируя таким образом дерево объёмов.

\subsection{Представление геометрии в САПР}\label{sec:geoCAD}

В BREP есть два типа понятий --- геометрические (<<точка>>, <<кривая>>, <<поверхность>>) и топологические (<<вершина>>, <<ребро>>, <<грань>>). <<Точка>> --- это тройка координат в некоторой системе координат. <<Кривая>> --- это уравнение, задающее множество точек, принадлежащих данной кривой. Кривую удобно описать с помощью параметрического уравнения от одной переменной. <<Поверхность>> --- это уравнение, задающее множество точек, принадлежащих данной поверхности. Соответственно, поверхность удобно описать с помощью параметрического уравнения от двух переменных. Топологические сущности задаются на базе геометрических. <<Вершина>> лежит в некоторой геометрической точке. <<Ребро>> лежит на некоторой геометрической кривой и ограничено двумя вершинами. Очевидно, что эти вершины должны принадлежать кривой, то есть и соответствующие геометрические точки должны принадлежать кривой. <<Грань>> лежит на некоторой поверхности и ограничена замкнутым циклом из рёбер. Также очевидно, что эти рёбра должны принадлежать поверхности, как и кривые, на которых они лежат, как и вершины и точки, ограничивающие эти рёбра. Замкнутая оболочка из граней с указанием внешних сторон этих граней ограничивает некоторую область пространства, называемую <<телом>>.

В соответствии с BREP параллелипипед (которому эквивалентен примитив box в CSG) задаётся следующим образом.\\
\textbf{Картинка и описание кратко}

Стоит однако отметить, что человек, создающий геометрическую модель в САПР, хотя и может выполнять построения в соответствии с базовыми принципами BREP, чаще всего применяет интуитивно понятные формообразования, из которых система точно формирует BREP модель в памяти ЭВМ, которая также необходима для получения триангулированной геометрии для визуализации на дисплее ЭВМ. Есть 4 базовых формообразования и 4 им обратных (с вычитанием) --- <<выдавливание>>, <<вращение>>, <<протягивание>> и <<тело по сечениям>>. Многие другие формообразования, такие как фаски и скругления, разрезы, отверстия, внутри на самом деле являются лишь вариациями перечисленных. Последовательность формообразований, выполненных пользователем для получения итоговой формы, сохраняется в виде дерева построения модели, напоминающего историю построения, но позволяющего навигацию и редактирование. Дерево часто доступно пользователю в основном рабочем окне интерфейса САПР. Однако бывают случаи, когда история построения теряется, например при передаче модели из одной САПР в другую. Таким образом, в результате работы инженера получается модель, описанная с помощью BREP, и во многих случаях имеющая также и дерево построения.

В инженерной практике принято проектировать и соответственно строить 3d-модели, объединяя в сборки детали и другие сборки. Отсюда вытекает, что во многих САПР, в том числе в CATIA~v5, существуют стандартные объекты, обозначающие детали и сборки. Например в САПР CATIA~v5 существует отдельный тип документа CATPart для детали и отдельный тип документа CATProduct для сборки. Внутри документа типа CATPart есть минимальный набор обязательных элементов --- 3 стандартные взаимноперпендикулярные плоскости в начале системы координат детали и главное тело детали, по умолчанию называемое PartBody. В документе типа CATProduct присутствует возможность добавлять в качестве дочерних компонентов либо документы CATPart либо другие документы CATProduct. В~\ref{sec:Builder} описывается, как соотносятся перечисленные сущности CATIA~v5 с понятиями геометрической подсистемы GEANT/ROOT.

Во многих САПР, в том числе и CATIA~v5 присутствует возможность так называемого контекстного редактирования компонентов. Это означает, что пользователь во время работы над сборкой в документе типа CATProduct, имеющей в качестве дочерних компонентов детали в файлах типа CATPart, может также редактировать детали, не переключая активный документ. Эта возможность широко используется в ``CATIA-GDML geometry builder'' --- большая часть работы выполняется в контексте единственного продукта, что с точки зрения пользователя аналогично работе над всей экспериментальной установкой.

\subsection{Представление геометрии в GEANT/ROOT}\label{sec:geoROOT}

Для описания геометрических форм в пакетах GEANT/ROOT применяется CSG. В качестве строительных блоков в CSG используются примитивы из списка реализованных в системе. Список примитивов включает в себя как относительно простые примитивы типа параллелипипед (box), сегмент цилиндра (tubs), сегмент конуса (cons), так и достаточно сложные, типа эллипсоида, параболоида, скрученных (twisted) примитивов. Принимая во внимание тот факт, что геометрия в GEANT/ROOT нужна для выполнения моделирования взаимодействия частиц с материалом, можно сказать, что примитив --- это объект, имеющий геометрическое представление и для которого реализовано решение геометрических задач, возникающих при моделировании. Среди таких геометрических задач можно отметить задачу нахождения расстояния до ближайшей границы примитива от некоторой точки внутри объёма, в одном заданном направлении или в любом возможном направлении. Эту задачу необходимо решать многократно в процессе проведения частицы для того, чтобы определить так называемый максимальный допустимый геометрический шаг. В результате моделирования физических процессов получается максимальный допустимый шаг из соображений физики. Для того чтобы собственно изменить координату частицы из этих двух шагов выбиратся минимальный.

Форма может быть описана как результат Булевой операции над примитивами или другими Булевыми операциями. Есть три Булевы операции --- объединение (union), вычитание (subtraction) и пересечение (intersection). Булевы операции позволяют задать практически любую геометрическую форму, имеющую границы из тех, что применяются в примитивах. При этом не требуется дополнительной реализации решения геометрических задач, т.к. удаётся комбинировать то, что реализовано в примитивах.

Таким образом наблюдается некоторая аналогия между BREP и CSG, заключающаяся в том, что в любом случае сложное тело или базовый примитив имеет некоторые границы, заданные аналитическими выражениями. Корни этой аналогии лежат в фундаментальной математике. Однако решающая разница заключается в том, что для примитива эти границы чётко определены и имеется лишь ограниченное число параметров, позволяющих изменять форму примитива.

Вторая составляющая геометрического представления в GEANT/ROOT это иерархия объёмов. Введём понятия логического и физического объёмов, формы и материала. Логический объём, или просто объём это базовый элемент для построения иерархии объёмов. Объём описывает непозиционированный объект и всё, что находится внутри него. Объём характеризуется формой и материалом. Форма --- это заданные с помощью CSG границы пространства, по методу, описанному выше. Материал включает в себя описание химического состава, плотности, и т.д. При помещение одного логического объёма в другой, например объёма $A$ в объём $B$, образуется так называемый физический объём, или узел, $B_1$, который обозначается взаимоотношение $A$ и $B$ как материнский-дочерний и характеризуется неоторой матрицей позиционирования $B$ внутри $A$.

\section{``CATIA-GDML geometry builder''}\label{sec:Builder}

``CATIA-GDML geometry builder'' (далее просто ``Builder'') представляет собой набор документов-шаблонов и макропрограмм для САПР CATIA~v5 вместе с настройками окружения и инструкциями к применению стандартных средств CATIA~v5. ``Builder'' ставит своей задачей упростить процесс создания CSG моделей с иерархией объёмов, напрямую совместимых с GEANT/ROOT.

Центральная идея ``Builder'' заключается в правилах соответствия сущностей CATIA~v5 и сущностей геометрии в GEANT/ROOT. Это соответствие делает возможным конвертацию MC-модели в CATIA~v5 в любой внешний файл с целью дальнейшего импорта в ROOT/GEANT. В качестве формата для обмена был выбран XML-подобный формат GDML (geometry description markup language), разработанный в CERN, для которого в GEANT4 и ROOT реализованы методы импорта и экспорта.

Вся геометрия установки создаётся в одном документа типа CATProduct. Объёму соответствует деталь, хранящаяся в файле типа CATPart. Форме соответствует главное тело детали, по умолчанию называемое PartBody. В CATIA не записывается описание материала так, как это принято в GEANT/ROOT, а сохраняется только имя материала в пользовательском параметре Material. Это возможно по той причине, что существует практика хранить описание материалов во внешнем файле или базе данных, считывать его перед выполнением моделирования и приписывать объёмам в соответствии с именами. Для обозначения физического объёма $B$ внутри $A$ в структуре документа, описывающего объём $A$, создаются тела Body.A.*, где * по умолчанию обозначает номер вхождения, но допускается запись любой идентифицирующей строки.

\textbf{Может быть частично перенести в описание методов описания геомтерии в GEANT/ROOT.}
Также в ``Builder'' предусмотрена возможность задания геометрии некоторыми продвинутыми методами, специфичными для GEANT/ROOT. В GEANT/ROOT существует тип объёvов, называемый Asssembly, который характеризуется тем, что он не имеет формы и материала. Практически объём типа Assembly является контейнером без границ, который объединяет свои дочерние объёмы, что особенно удобно как минимум в двух случаях. Во-первых, если необходимо многократно позиционировать группу объёмов, которую невозможно охватить простой формой. Во-вторых, если преобразование координат при позиционировании одного или группы объёмов имеет сложную структуру и удобно представить его как суперпозицию двух преобразований. Как частный случай можно упомянуть ситауцию, когда какой-либо параметр преобразования является параметром модели (см. секцию~\ref{sec:Parameterization})

Один из плюсов ``Builder'' заключается в том, что пользователю предоставляется возможность работать с полноценной инженерной моделью и MC-моделью в одной и той же среде, имеющей широкие возможности для анализа и редактирования геометрии. ``Builder'' не ставит своей задачей перевод модели из одного геометрического представления в другой, но значительно ускоряет процесс создания одной геометрии, на основе другой. На нашей практике это означает, что ??? забыл мысль

``Builder'' включает в себя файлы, в которых специальным образом построены примитивы GEANT/ROOT, позволяющие пользователю при построении MC-модели не вникать в подробности реализации, а использовать их практически как и в процессе создания геометрии средствами ROOT или GEANT. ``Builder'' также включает в себя макропрограммы для CATIA~v5, которые также ставят своей задачей сделать процесс построения геометрии в ``Builder'' максимально похожим на процесс построения геометрии в GEANT или ROOT. Основной макрос --- это конвертер ``CATIA2GDML'', который проецирует дерево построения модели в CATIA в GDML файл. Также разработан обратный конвертер ``GDML2CATIA'' для импорта GDML файлов.

Целевая аудитория ``CATIA-GDML geometry builder'' --- физики, владеющие CATIA~v5 на базовом уровне, и инженеры, продвинутые пользователи САПР, изучившие способ представления геометии в GEANT/ROOT хотя бы на теоретическом уровне.

\textbf{Есть опыт, который показывает, что для достижения такого уровня как физикам, так и инженерам, достаточно прохождения двухнедельного курса.}

Предлагается новый алгоритм работы, в котором ``Builder'' используется как многофункциональный инструмент. Описанный ниже алгоритм сформулирован на успешном опыте разработки CBM RICH на протяжении 3 (4) лет.

Задача создания и поддержания актуальной MC-модели поручается ответственному человеку, владеющему CATIA, GEANT/ROOT и ``CATIA-GDML geometry builder''.

В зависимости от того, какая информация и в каких файлах имеется к началу работы, алгоритм немного различается.
Если разработка ведётся в нуля и нет никаких данных в ЭВМ, что возможно, например, когда проект находится на таком этапе, когда нужно выполнить грубое моделирование, показывающее принципиальную возможность реализации, то наиболее оптимальный способ --- сразу строить MC-модель в CATIA средствами ``Builder''.
Если, скажем, проект находится на раннем этапе разработки и уже имеется какая-то приблизительная САПР модель, то рекомендуется импортировать её стандартными средствами CATIA, чтобы затем на её основе построить MC-модель в CATIA в автоматизированном режиме с помощью средств ``Builder''. Инженерную геометрию можно импортировать практически из любой САПР, например с помощью широко распространённого формата STEP. 
Третий распространённый случай это когда уже имеется некоторая MC-модель в конечной системе моделирования. Как в GEANT4, так и в ROOT имеется стандартная возможность экспортировать геометрию в GDML файл без потери информации. Эту возможность могут наследовать все дочерние пакеты (как FairRoot и далее CbmRoot), но для этого необходимо явно активировать функциональность GDML. В этом случае можно импортировать модель в CATIA в MC-формате, однако иногда требуются некоторые дополнительные ручные операции после импорта. Они выполняются однократно и лишь делают структуру документа более оптимальной, но не изменяют геометрию.

Во всех этих алгоритмах, независимо от типа и количества исходных данных, получаются файлы CATIA в формате ``Builder'', которые в дальнейшем будут являться основными (первичными) файлами для получения рабочей MC-модели в экспериментальном пакете, которым в случае CBM RICH является CbmRoot. Модель из CATIA экспортируется в GDML файл, который не требует каких-либо последующих изменений в структуре. Для достижения этого условия была проведена огромная работа по мере разработки MC-модели CBM RICH. Допускается и даже рекомендуется текстовое редактирование GDML файла, но только для изменения значений параметров в define секции у параметризованных моделей. Затем, по желанию коллаборации, GDML файл может быть конвертирован в бинарный ROOT-файл, который содержит геометрию, которую невозможно редактировать. Это защищает модель от случайных изменений, что особенно актуально в случае параметризованных моделей. Соответственно, если требуется изменить значения параметров, пользователь может отредактировать GDML файл и экспортировать в новый ROOT файл. Практика показывает, что в случае, если требуется множество файлов с MC-геометрией, то обязательно нужно писать комментарии --- либо в самом GDML файле, либо в текстовом файле рядом с GDML/ROOT файлом. Обычно в коллаборации вводят правила именования файлов.

\subsection{Макропрограммы для CATIA~v5}\label{sec:Macros}

Макропрограммы для CATIA~v5 написаны на VBA с применением CATIA API. Все макропрограммы, кроме \macroname{AddShape} и \macroname{Poly}, доступны пользователю в режиме работы над сборкой. В CATIA различают открытый документ (верхний в дереве в текущем окне), активный документ (синий), выделенный объект (оранжевый) и рабочий объект (подчёркнутый). Пользователь может выполнить все операции, необходимые для получения MC-модели, самостоятельно без применения макропрограмм, но в этом случае велика вероятность упустить какой-либо шаг, что приведёт к ошибке, которую сложно диагностировать.

В MC-модели в CATIA есть строгие правила именования. Применение макросов избавляет пользователя от необходимости контролировать имена объектов в документах. Все имена, сгенерированные при использовании ``Buider'' не конфликтуют между собой и позволяют получить корректный GDML файл на выходе. Практически везде пользователь имеет право изменять суффиксы, не изменяя основного названия, несущего информацию о типе объекта --- формообразования, тела, и т.д. Однако в редких случаях суффикс имеет решающее значение, как например в именах поворотах (напр, ``Rotate.X'') суффикс несёт информацию о оси поворота.

В процессе разработки ``Builder'' был выработан стандартный алгоритм создания геометрии. Первый шаг --- создание нового документа типа CATProduct, который в дальнейшем будет единственным продуктом, и его сохранение на диск. Этот продукт будет представлять модель всей экспериментальной установки. Второй этап --- наполнение продукта описанием объёмов без описания взаимосвязей между ними. Для этого используется макрос \macroname{AddNewPart}, который автоматически открывает в отдельном окне новый документ типа CATPart, сформированный из специального шаблона и соответствующий создаваемому объёму. Система переходит в режим редактирования детали, где доступны только два макроса \macroname{AddShape} и \macroname{Poly} для создания формы объёма. Здесь же можно и задать имя материала объёма. По окончании редактирования нового объёма в отдельном окне пользователь должен сохранить активный документ и закрыть это окно. CATIA при этом возвращается к редактированию продукта. После того, как созданы объёмы, заданы формы и, возможно, имена материалов, алгоритм подразумевает задание иерархии объёмов, то есть позиционирование одних объёмов в других. Для этого в ``Builder'' существует целый ряд макропрограмм для создания различных типов взаимосвязей --- \macroname{Inserter}, \macroname{ArrayMaker}, \macroname{Replica}. После того, как выполнено размещение дочернего объёма $A$ в материнском объёме $B$, пользователь может указать поворот и сдвиг, задающие матрицу позиционирования $A$ в $B$. Для упрощения расчётов в некоторых случаях очень удобно применять макропрограммы \macroname{PointToPointAligner} (\macroname{Pt2PtAligner}), \macroname{Mover} и \macroname{Measure}. Для удобного редактирования материалов всех объёмов был разработан менеджер материалов \macroname{MaterialsManager}, который обычно имеет смысл вызывать перед экспортом для проверки ранее заданных имён материалов, либо назначения новых. Также перед экспортом рекомендуется проверить модель на наличие ошибок с помощью макроса \macroname{Checker}. В конце выполняется экспорт макросом \macroname{CATIA2GDML}. Отдельно стоят макропрограммы \macroname{Duplicator} для создания множественных идентичных, но не связанных, параметризованных подсборок и обратный конвертер \macroname{GDML2CATIA} для импорта GDML файла.

Для комфортной работы с ``Builder'' в поставке также имеются файлы для настройки окружения CATIA. Использования окружения в принципе не обязательно, но часть функционала зависит от путей к файлам, которые прописаны в переменных окружения, поэтому настоятельно рекомендуется перед использованием ``Builder'' выполнить настройку, следуя инструкции, поставляемой в пакете.

\subsubsection{AddNewPart}\label{sec:AddNewPart}

Данная макропрограмма автоматизирует создание нового документа типа CATPart на основе шаблона, содержащего необходимые элементы --- публикация главного тела детали, называемая PartBody, пользовательский параметр под названием Material со значением по умолчанию ???. Также для удобства погашены стандартные плоскости.

\subsubsection{AddShape}\label{sec:AddShape}

\macroname{AddShape} используется для создания примитивов, в случае необходимости вместе с поворотами и сдвигом. Макропрограмма играет роль интерфейса между пользователем и файлами примитивов. При запуске макроса выводится окно со списком доступных примитивов, по нажатии на кнопку ``создать'' в рабочее тело детали вставляется выбранный примитив со значениями параметров по умолчанию. Если на форме графического интерфейса выбраны флаги создания поворотов и сдвига, то создаются соответствующие формообразования.

\subsubsection{Poly}\label{sec:Poly}

В силу ограничений CATIA нет возможности представить полипримитивы (polycone и polyhedra) с помощью тех же средств, что и остальные примитивы, поэтому для них была разработана специальная структура дерева и правила именования. Для автоматизации построения полипримитивов в соответствии с этой структурой предоставляется макрос \macroname{Poly}. Секции поликонуса представлены стандартными конусами. В случае polyhedra для представления секции используется hedra --- специальный примитив, не поддерживаемый GEANT/ROOT.

\subsubsection{Inserter}\label{sec:Inserter}

Макрос \macroname{Inserter} --- это инструмент для помещения одного выбранного объёма в другой. Также можно сказать, что \macroname{Inserter} создаёт физический объём, задающий связь материнский-дочерний между двумя существующими логическими объёмами. \macroname{Inserter} --- возможно, самый используемый макрос, в результате работы которого в документе типа CATPart, представляющем материнский объём, создётся тело с именем ``Body.B.*'', где $B$ --- имя дочеhнего объёма. Внутри этого тела имеется ссылка на публикацию PartBody документа типа CATPart, представляющего объём $B$, и элементы преобразования типа Rotate и Translate --- три поворота и сдвиг, задающие матрицу позиционирования $B$ внутри $A$.

\subsubsection{ArrayMaker}\label{sec:ArrayMaker}



\subsubsection{Replica}\label{sec:Replica}

\subsubsection{PointToPointAligner}\label{sec:PointToPointAligner}

\subsubsection{Mover}\label{sec:Mover}

\subsubsection{Measure}\label{sec:Measure}

\subsubsection{MaterialsManager}\label{sec:MaterialsManager}

\subsubsection{Checker}\label{sec:Checker}

\subsubsection{CATIA2GDML}\label{sec:CATIA2GDML}

\subsubsection{Duplicator}\label{sec:Duplicator}

\subsubsection{GDML2CATIA}\label{sec:GDML2CATIA}

\macroname{GDML2CATIA} выполняет процедуру, обратную \macroname{CATIA2GDML} --- проецирует GDML файл на дерево модели CATIA~v5. Отличительной особенностью является то, что в ``Builder'' есть возможность задавать линейные и круговые массивы --- многократные вхождения дочернего объёма в материнский, позиционированные с некоторым шагом вдоль линейной или круговой оси соответственно. В CAITA для массивов применяется соответствующее стандартное формообразование pattern. Такая возможность отсутствует в GDML, поэтому при экспорте из CATIA в GDML выполняется расчёт поворотов и сдвигов для каждого элемента массива и они представляются как отдельные, независимые дочерние объёмы. Таким образом при конвертации в обратном направлении, из GDML в CATIA, невозможно восстановить массив. Следовательно, одна из немногих (единственная) операций, которые необходимо совершать после импорта геометрии из GDML в CATIA --- ручной перевод множества дочерних объёмов в массив. Обычно это очень простая процедура, 



\subsubsection{Параметризация}\label{sec:Parameterization}

Одна из наиболее важных возможностей ``CATIA-GDML geometry builder'' --- это возможность создания параметризованных геометрических MC-моделей. У параметризованной модели имеются входные параметры и формулы, задающие зависимость между этими входными параметрами и внутренними переменными, такими как параметры примитивов, значения поворотов и смещений. Данная концепция хорошо ложится на методы работы с геометрией в САПР, особенно CATIA~v5. Также параметризация поддерживается форматом GDML и импортёрами GEANT(?) и ROOT.

В модели CATIA~v5 можно вводить пользовательские параметры как в документах типа CATProduct, так и в документах типа CATPart. Причём сборка в CATProduct файле может иметь свои пользовательские параметры и формулы, а дочерние компоненты в CATPart файлах --- свои. Обязательное требование ``Builder'' таково, что все параметры и формулы должны находиться в верхнем продукте. CATIA~v5 позволяет задавать зависимости между любыми параметрами, в том числе внутренними, не являющимися пользовательскими, однако для успешного экспорта в GDML файл формула должна в левой части иметь параметр примитива или угол поворота или значение сдвига, а в парвой части --- формулу только над пользовательскими параметрами. Пользовательский параметр CATIA~v5, экспортируемый в переменную в GDML должен обязательно иметь безразмерный тип Real. В связи с этим имеются правила оформления формул и приведения единиц измерения. Также имеется стандартная переменная DEGtoRAD для перевода значения улгов из градусов в радианы.

На выходе получается GDML файл, у которого в define секции есть тэги variable, обозначающие входные параметры модели со значениями. При импорте параметризованной геометрии из GDML в ROOT все значения внутренних переменных рассчитываются в соответствии с формулами по значениям входных параметров и параметризация теряется. Следовательно значения входных параметров должны задаваться пользователем непосредственно в GDML файле перед импортом в конечную систему.

\section{Избранные подробности реализации ``CATIA-GDML geometry builder''}

Каждый макрос ``Builder'' --- это VBA проект, который хранится в отдельном catvba файле. Проект состоит из трёх разделов --- элементы графического интерфейса (формы), модули и модули классов. Большинство макросов ``Builder'' написано в соответствии с идеологией структурного программирования, без применения классов, и разделение на модули выполнено из соображений читаемости кода. Обычно в отдельный модуль выносился функционал, объединённый некоторой задачей. Так, например, во многих макросах имеется модуль ??? (имя) для продвинутой работы со строками, модуль ??? (имя) для ??? (задача). В некоторых случаях естественным образом требовалось использовать классы. Так, например, был реализован класс матрицы с методами нахождения углов поворота, который использовался в ??? и более подробно описан в~\ref{sec:Matrices}.

\subsection{Работа с матрицами позиционирования в ``CATIA-GDML geometry builder''}\label{sec:Matrices}



\section{Применение ``CATIA-GDML geometry builder'' к CBM RICH}\label{sec:RICHgeo}

Значительная часть работы над ``Builder'' выполнялась при поддержке группы CBM RICH, поэтому самая сложная MC-модель построенная с помощью ``Builder'' это CBM RICH. Построенная за несколько итераций модель имеет достаточно сложную иерархию и характеризуется высокой степенью подробностей.




Для того, чтобы GDML без проблем импортировался в CbmRoot было написано дополнение.
