\section{Сравнение представления геометрии в GEANT/ROOT и САПР}\label{sec:ROOTvsCAD}

Разница между двумя способами описания геометрической информации в САПР и пакетах моделирования прохождения частиц через материал GEANT/ROOT заключается в двух пунктах. Во-первых, отличается способ задания геометрических форм. В САПР применяется граничное представление (BREP), для описания которого используются понятия типа <<поверхность>>, <<грань>>, <<ребро>>, <<кривая>>, и за которыми стоят соответствующие уравнения, описывающие эти объекты в пространстве. В GEANT/ROOT применяется конструктивная твердотельная геометрия (CSG), которая оперирует понятиями <<примитив>> и <<Булева операция>>. Очевидно, что и за этими объектами также стоят конкретные уравнения, описывающие кривые и поверхности, однако есть существенное различие описанное ниже. Во-вторых, отличается способ задания взаимоотношения форм в пространстве. В САПР, по аналогии с тем, как человек воспринимает окружающий мир, присутствует некоторое бесконечное окружающее пространство без материала, а все предметы находятся в этом пространстве. Невозможна такая ситуация, чтобы один объект находился внутри другого --- в таком случае подразумевается, что во втором есть соответствующая полость, освобождающая место под первый объект. В GEANT/ROOT для описания взаимоотношения форм используется иерархия объёмов. Это объясняется тем, что такой метод более удобен для описания геометрии, где главной задачей является однозначное задание материала в каждой точке пространства. Вводится понятие объёма --- сущности, имеющей форму и материал. Из всех объёмов выбирается один, называемый объёмом верхнего уровня, а остальные помещены либо в него, либо в какой-то другой, формируя таким образом дерево объёмов.

\subsection{Представление геометрии в САПР}\label{sec:geoCAD}

В BREP есть два типа понятий --- геометрические (<<точка>>, <<кривая>>, <<поверхность>>) и топологические (<<вершина>>, <<ребро>>, <<грань>>). <<Точка>> --- это тройка координат в некоторой системе координат. <<Кривая>> --- это уравнение, задающее множество точек, принадлежащих данной кривой. Кривую удобно описать с помощью параметрического уравнения от одной переменной. <<Поверхность>> --- это уравнение, задающее множество точек, принадлежащих данной поверхности. Соответственно, поверхность удобно описать с помощью параметрического уравнения от двух переменных. Топологические сущности задаются на базе геометрических. <<Вершина>> лежит в некоторой геометрической точке. <<Ребро>> лежит на некоторой геометрической кривой и ограничено двумя вершинами. Очевидно, что эти вершины должны принадлежать кривой, то есть и соответствующие геометрические точки должны принадлежать кривой. <<Грань>> лежит на некоторой поверхности и ограничена замкнутым циклом из рёбер. Также очевидно, что эти рёбра должны принадлежать поверхности, как и кривые, на которых они лежат, как и вершины и точки, ограничивающие эти рёбра. Замкнутая оболочка из граней с указанием внешних сторон этих граней ограничивает некоторую область пространства, называемую <<телом>>.

В соответствии с BREP параллелипипед (которому эквивалентен примитив box в CSG) задаётся следующим образом.\\
\textbf{Картинка и описание кратко}

Стоит однако отметить, что человек, создающий геометрическую модель в САПР, хотя и может выполнять построения в соответствии с базовыми принципами BREP, чаще всего применяет интуитивно понятные формообразования, из которых система точно формирует BREP модель в памяти ЭВМ, которая также необходима для получения триангулированной геометрии для визуализации на дисплее ЭВМ. Есть 4 базовых формообразования и 4 им обратных (с вычитанием) --- <<выдавливание>>, <<вращение>>, <<протягивание>> и <<тело по сечениям>>. Многие другие формообразования, такие как фаски и скругления, разрезы, отверстия, внутри на самом деле являются лишь вариациями перечисленных. Последовательность формообразований, выполненных пользователем для получения итоговой формы, сохраняется в виде дерева построения модели, напоминающего историю построения, но позволяющего навигацию и редактирование. Дерево часто доступно пользователю в основном рабочем окне интерфейса САПР. Однако бывают случаи, когда история построения теряется, например при передаче модели из одной САПР в другую. Таким образом, в результате работы инженера получается модель, описанная с помощью BREP, и во многих случаях имеющая также и дерево построения.

В инженерной практике принято проектировать и соответственно строить 3d-модели, объединяя в сборки детали и другие сборки. Отсюда вытекает, что во многих САПР, в том числе в CATIA~v5, существуют стандартные объекты, обозначающие детали и сборки. Например в САПР CATIA~v5 существует отдельный тип документа CATPart для детали и отдельный тип документа CATProduct для сборки. Внутри документа типа CATPart есть минимальный набор обязательных элементов --- 3 стандартные взаимноперпендикулярные плоскости в начале системы координат детали и главное тело детали, по умолчанию называемое PartBody. В документе типа CATProduct присутствует возможность добавлять в качестве дочерних компонентов либо документы CATPart либо другие документы CATProduct. В~\ref{sec:Builder} описывается, как соотносятся перечисленные сущности CATIA~v5 с понятиями геометрической подсистемы GEANT/ROOT.

Во многих САПР, в том числе и CATIA~v5 присутствует возможность так называемого контекстного редактирования компонентов. Это означает, что пользователь во время работы над сборкой в документе типа CATProduct, имеющей в качестве дочерних компонентов детали в файлах типа CATPart, может также редактировать детали, не переключая активный документ. Эта возможность широко используется в ``CATIA-GDML geometry builder'' --- большая часть работы выполняется в контексте единственного продукта, что с точки зрения пользователя аналогично работе над всей экспериментальной установкой.

\subsection{Представление геометрии в GEANT/ROOT}\label{sec:geoROOT}

Для описания геометрических форм в пакетах GEANT/ROOT применяется CSG. В качестве строительных блоков в CSG используются примитивы из списка реализованных в системе. Список примитивов включает в себя как относительно простые примитивы типа параллелипипеда (box), сегмента цилиндра (tubs), сегмента конуса (cons), так и достаточно сложные, типа эллипсоида, параболоида, скрученных (twisted) примитивов. Принимая во внимание тот факт, что геометрия в GEANT/ROOT нужна для выполнения моделирования взаимодействия частиц с материалом, можно сказать, что примитив --- это объект, имеющий геометрическое представление и для которого реализовано решение геометрических задач, возникающих при моделировании. Среди таких геометрических задач можно отметить задачу нахождения расстояния до ближайшей границы примитива от некоторой точки внутри объёма, в одном заданном направлении или в любом возможном направлении. Эту задачу необходимо решать многократно в процессе проведения частицы для того, чтобы определить так называемый максимальный допустимый геометрический шаг. В результате моделирования физических процессов получается максимальный допустимый шаг из соображений физики. Для того чтобы собственно изменить координату частицы из этих двух шагов выбиратся минимальный.

Форма может быть описана как результат Булевой операции над примитивами или другими Булевыми операциями. Есть три Булевы операции --- объединение (union), вычитание (subtraction) и пересечение (intersection). Булевы операции позволяют задать практически любую геометрическую форму, имеющую границы из тех, что применяются в примитивах. При этом не требуется дополнительной реализации решения геометрических задач, т.к. удаётся комбинировать то, что реализовано в примитивах.

Таким образом наблюдается некоторая аналогия между BREP и CSG, заключающаяся в том, что в любом случае сложное тело или базовый примитив имеет некоторые границы, заданные аналитическими выражениями. Корни этой аналогии лежат в фундаментальной математике. Однако решающая разница заключается в том, что для примитива эти границы чётко определены и имеется лишь ограниченное число параметров, позволяющих изменять форму примитива.

Вторая составляющая геометрического представления в GEANT/ROOT это иерархия объёмов. Введём понятия логического и физического объёмов, формы и материала. Логический объём, или просто объём это базовый элемент для построения иерархии объёмов. Объём описывает непозиционированный объект и всё, что находится внутри него. Объём характеризуется формой и материалом. Форма --- это заданные с помощью CSG границы пространства, по методу, описанному выше. Материал включает в себя описание химического состава, плотности, и т.д. При помещение одного логического объёма в другой, например объёма $A$ в объём $B$, образуется так называемый физический объём, или узел, $B_1$, который обозначается взаимоотношение $A$ и $B$ как материнский-дочерний и характеризуется неоторой матрицей позиционирования $B$ внутри $A$.