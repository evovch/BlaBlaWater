\chapter{Эксперимент CBM на FAIR}\label{sec:secCbm}

\section{Физическая программа CBM}

\section{Экспериментальная установка CBM}\label{sec:secCbmSetup}

\subsection{Вершинный микродетектор MVD}\label{sec:secMVD}

\subsection{Кремниевая трекинговая система STS}\label{sec:secSTS}

From CBM STS TDR:

The STS is operated in a thermal enclosure that keeps the sensors at a temperature of about \SI{-5}{\degreeCelsius}. The heat dissipated in the read-out electronics is removed by a $CO_{2}$ cooling system. The mechanical structure of the detector system including the service and signal connections is designed such that single detector ladders can be exchanged without disconnecting and removing more than one detector station.

Задача кремниевой трековой системы --- измерение траекторий и импульсов заряженных частиц, вылетающих из точки взаимодействия пучка тяжёлых ионов с мишенью. Для выполнения физической программы CBM --- наблюдения редких \textbf{явлений} --- необходима частота взаимодействий до 10~МГц, при том что в одном взаимодействии будет рождаться до 1000 заряженных частиц. Реконструкция треков должна выполняться с эффективностью порядка 95\% и разрешением по импульсу порядка $\Delta p / p = 1\%$. Для удовлетворения перечисленных требований STS должна состоять из 8~слоёв кремниевых микростриповых сенсоров, расположенных внутри поля от дипольного магнита на растоянии от 30~см до 100~см от точки взаимодействия вниз по пучку с шагом 10~см.
Сенсоры будут монтироваться на легкую механическую опору в виде карбоновых ферм. Считывание будет осуществляться по многоканальным микро-кабелям самотриггирующейся электроникой, расположенной по краям станций вместе с линиями охлаждения и другими инфраструктурными подсистемами. Микро-кабели будут выполнены из sandwiched polyimide-Aluminum layers толщиной несколько десятков мкм.
%The microstrip sensors will be double-sided with a stereo angle of \SI{7.5}{\degree}
Микростриповые сенсоры будут двухсторонними \todo, шаг между стрипами $58 \mu$м, длина стрипов от 20~до~60~мм, а толщина кремния $300 \mu$м. По текущим оценкам максимальная неионизирующая доза в CBM для сенсоров, расположенных ближе всего к пучку, не будет превышать $10^{14}$ $n_{eq}$ см$^{-2}$. 

\subsection{Детектор черенковских колец RICH}\label{sec:secRICH}

\subsection{Мюонная система MUCH}\label{sec:secMUCH}

From CBM MUCH TDR:

The MuCh system is designed to identify muon pairs which are produced in high-energy heavy-ion collisions in the beam energy range from~4~to~40~AGeV. The measurement of lepton pairs is a central part of the CBM research program, as they are very sensitive diagnostic probes of the conditions inside the fireball. At low invariant masses, dileptons provide information on the in-medium modification of vector mesons which is a promising observable for the restoration of chiral symmetry. At intermediate invariant masses, the dilepton spectrum is dominated by thermal radiation from the fireball reflecting its temperature. At invariant masses around $ 3 GeV/c^{2} $, dileptons are the appropriate tool to study the anomalous charmonium suppression in the deconfined phase. In the CBM experiment both electrons and muons will be measured in order to obtain a consistent and comprehensive picture of the dilepton physics.

The experimental challenge for muon measurements in heavy-ion collisions at FAIR energies is to identify low-momentum muons in an environment of high particle densities. The CBM strategy is to track the particles through a hadron absorber system, and to perform a momentum-dependent muon identification. This concept is realized by an instrumented hadron absorber, consisting of staggered absorber plates and tracking stations. The hadron absorbers vary in material and thickness, and the tracking stations consist of detector triplets based on different technologies. The MuCh system is placed downstream of the dipole magnet hosting the Silicon Tracking System (STS) which determines the particle momentum. In order to reduce the number of muons from pion and kaon weak decays, the absorber/detector system has to be as compact as possible.

The MuCh system will be built in stages which are adapted to the beam energies available. Within the FAIR modularized start version the SIS100 ring will provide heavy ion beams with energies up to 14~AGeV, and proton beams up to 29~GeV. The first two versions of MuCh (SIS100-A and SIS100-B) will comprise of~3~and~4 stations suitable for the measurement of low-mass vector mesons in $ A + A $ collisions at 4-6~AGeV and 8-14~AGeV, respectively. The third version of the MuCh system (SIS100-C) will be equipped with an additional iron absorber of 1~m thickness in order to be able to identify charmonium at the highest SIS100 energies. The absorber slices will be built only once so that they could be rearranged properly to obtain required absorber thicknesses. Once SIS300 is operational, we will upgrade the MuCh system further by inserting additional absorbers and detector stations for the measurement of low-mass vector mesons and charmonium at beam energies above 14~AGeV (MuCh versions SIS300-A and SIS300-B).

\subsection{Детектор переходного излучения TRD}\label{sec:secTRD}

\subsection{Время-пролётный детектор TOF}\label{sec:secTOF}

\subsection{Электромагнитный калориметр ECAL}\label{sec:secECAL}

\subsection{Детектор PSD}\label{sec:secPSD}

\section{CBM RICH поподробней}