\chapter*{Введение}\label{sec:secIntro}
\addcontentsline{toc}{chapter}{Введение}

\section*{Intro 0.1}
\textbf{Фазовая диаграмма (что такое и почему нужно ее изучать), физическая программа СBM, и сравнение с другими экспериментами в этой области (по доступной области фазовой диаграммы и наблюдаемым, жесткость требований для детекторов --- в разделе актуальность), подводка к теме работы. Этот раздел не более 5-6 страниц.}

Одна из важнейших задач современной физики --- это исследование уравнения состояния ядерного вещества. Для решения этой задачи необходимо определить границы существования различных фаз и описать их свойства. Совокупность теоретических представлений по данному вопросу отображается на фазовой диаграмме, см.~\figref{} (рисунок надо выбрать такой, где видны области, доступные разным экспериментам). Здесь по одной оси отложен барионный химический потенциал, связанный с плотностью барионов, а по другой --- температура. Актуальные экспериментальные исследования направлены на установлению границы между барионной материей и кварк-глюонной плазмой (КГП)~\cite{}, локализации критической точки и исследованию свойств материи в указанных областях фазовой диаграммы.

\todo \textbf{Здесь упоминаются эксперименты, аналогичные CBM.}

Наиболее важны в этой области действующие эксперименты STAR и ALICE.

Какая часть фазовой диаграммы и какие наблюдаемые, STAR, ALICE, MPD.
Теперь еще что надо:
для Стар и Алиса в наиболее жестком режиме найти
ионы до (самый тяжелый)
энергии до
частота взаимодействия до
число вторичных частиц на событие до
если получится (можете написать кому-то и спросить) максимальная угловая плотность частиц.

ЗДЕСЬ о том, какую область фазовой диаграммы и какими наблюдаемыми зондирует каждый эксперимент.

Из строящихся экспериментов наиболее важны NICA MPD,(область, наблюдаемые) и CBM.

Физическая программа CBM состоит в следующем.


В экспериментах в ЦЕРНе и Брукхейвенской национальной лаборатории поиск критической точки осуществляется только посредством регистрации спектральных характеристик потоков вторичных частиц нескольких типов, рождающихся в большом количестве. Эксперименты FAIR, благодаря высокой интенсивности первичных пучков, открывают дополнительную возможность регистрировать редкие события со сканированием обширной области фазовой диаграммы по энергиям частиц. В частности планируется впервые непосредственно исследовать признаки возникновения ``огненного шара'' (fireball) --- области ядерной материи, в которой произошёл переход от барионной фазы к кварк-глюонной фазе, --- с помощью регистрации короткоживущих векторных мезонов, распадающихся на дилептонные пары.

Диапазон энергий FAIR 2--35~\GeVperNucl для ионов золота хорошо подходит для проведения экспериментов в области фазовой диаграммы с высокими плотностями ядерной материи, превосходящими нормальную плотность в 8--10 раз (уже было).

% https://www.bnl.gov/npp/docs/tribble090712/Vigdor_RHIC_overview_rev2.pdf - слайд 7

RHIC планирует ``даунгрейд'' для выполнения скана по фазовой диаграмме, но \todo всё-равно не заменит FAIR.
\todo \textbf{Здесь описать то, как соотносится FAIR с другими ускорителями, на которых выполняются или планируются эксперименты аналогичные CBM. Здесь появляется фазовая диаграмма, LHC, RHIC, NICA. При этом о самих экспериментах, аналогичных CBM, разговор идёт чуть дальше после описания физической программы CBM. Может быть оставить как есть в следующей секции и ничего тут уж не писать?}

\todo \textbf{В разных источниках числа расходятся. Где-то 35, где-то 45...}

Физическая программа CBM нацелена на исследование свойств сверхплотной барионной материи, образующейся в ядро-ядерных столкновениях при энергии пучка от~2~до~45~\GeVperNucl. CBM проектируется с учётом необходимости справляться с измерением высокой статистики адронных, лептонных и фотонных проб в большом аксептансе. Физическая программа включает в себя множество наблюдаемых, среди которых:

\begin{itemize}
\item выход и коллективный поток странных и очарованных адронов; ожидается что они отразят процесс становления деконфайнмента;
\item коллективный поток адронов, который особенно чувствителен к уравнению состояния ядерного вещества на ранних стадиях реакций;
\item производство частиц при пороговых энергиях (странность на SIS100 и очарованность на SIS300), которое может нести важную информацию об уравнении состояний ядерной материи;
\item нестатистические отклонения от события к событию различных параметров (выходы частиц, отношения выходов), связанные с сохранением квантовых чисел (барионных, заряда, странности), которые могут служить сигналом о критической точке КХД;
\item изменение адронных масс в среде, в частности изменение \todo, которые предоставят ценную информацию о внутренних процессах при ожидаемом восстановлении киральной симметрии в плотной барионной материи.
\end{itemize}

\todo \textbf{плоскость реакции - ?}

Высокая интенсивность пучка и продолжительная его доступность позволят CBM впервые измерять редкие пробы, такие как очарованные адроны и лёгкие векторные мезоны (с помощью дилептонных распадов), в области энергий, предоставляемых FAIR.

Экспериментальная задача CBM --- измерять перечисленные наблюдаемые в A+A, p+A, p+p столкновениях как функцию энергии столкновения и размера системы с высокой точностью и статистикой, а также искать нарушения непрерывностей, которые могут служить сигналом о фазовом переходе первого уровня. Данная физическая программа будет выполняться измерением ядерных столкновений при экстремально высоких частотах взаимодействия.

Данная работа посвящена методически разработкам для детектора RICH, учавствующего в измерении таких наблюдаемых как low mass vector mesons and J/phi по диэлектронному каналу.

\section*{Актуальность работы:}

Современные эксперименты в области физики высоких энергий и, особенно, столкновения релятивистских тяжелых ионов выдвигают жёсткие требования к проектным решениям при создании установок по причине высокой загрузки, высокой плотности потока частиц и высокой радиационной нагрузки, сложности и многопараметричности моделей, описывающих изучаемые эффекты, тонкости последних и высоких фонов.

Например, (рассказать c конкретными цифрами про загрузки, наблюдаемые (измеримые величины) и фоны в ряде экспериментов, в частности, flow measurements и rare probes в CBM, что-то обязательно про СТАР, АЛИСУ и какой-нибудь еще эксперимент с фиксированной мишенью, подчеркнуть особенности и жесткость требований CBM).

Все эти факторы

Требуются совершенные методы моделирования детекторов с высоким уровнем детализации и возможностью выполнения нескольких итераций расчетов, а также разработка новых систем сбора данных, адекватных современному аппаратному обеспечению и ожидаемым потокам информации.

Кроме того, необходимы интенсивные исследования прототипов создаваемых детекторов.

В настоящей диссертации обсуждаются все три перечисленных аспекта (развернуть на 1-2 абзаца) в применении, в первую очередь, к детектору Черенковских колец эксперимента CBM (далее CBM RICH).

\section*{Цели:}

\begin{itemize}
\item{разработать инструментарий для облегчения создания детальных геометрических моделей, предназначенных для таких сред Монте-Карло моделирования прохождения частиц через вещество, как Geant4 и ROOT, а также для обмена геометрической информацией между этими средами и САПР CATIA~v5;}
\item{создать гибкое и точное описание детектора CBM RICH в среде CbmRoot и осуществить на основе этого описания оптимизацию конструкции и компоновки данного детектора;}
\item{создать ПО для испытания прототипа системы считывания и сбора данных детектора CBM RICH в составе полнофункционального прототипа указанного детектора на пучковых тестах;}
\item{провести исследование свойств прототипа системы считывания и сбора данных детектора CBM RICH на основе результатов пучковых тестов и измерений на лабораторном стенде.}
\end{itemize}

\section*{Научная новизна и практическая ценность работы:}

\begin{enumerate}
\item Разработана схема отображения иерархии геометрии, используемой в моделировании транспорта частиц методом Монте-Карло (МК), на дерево построений САПР CATIA~v5.
\item В среде CATIA~v5 создан набор шаблонов для примитивов и сущностей конструктивной твердотельной геометрии, принятой в системах МК моделирования детекторов.
\item Создан набор инструментов для полуавтоматического построения детальной МК геометрии на основе САПР модели и быстрого обмена геометрией между САПР CATIA~v5 и пакетами МК моделирования GEANT и ROOT.
\item Выполнены беспрецедентно точные параметризованные описания ряда приборов и детекторов в средах МК моделирования.
\item На основе детального параметризованного описания геометрии CBM~RICH выполнена оптимизация компоновки детектора. 
\item Собран прототип системы считывания и сбора данных детектора CBM~RICH.
\item Разработано программное обеспечение для приема, упаковки и передачи бестриггерного потока данных с прототипа системы считывания и сбора данных с частотой до 20~МГц.
\item Разработано программное обеспечение для калибровки точного времени и относительных задержек каналов в потоке данных с детектора CBM RICH.
\item Разработано программное обеспечение для построения событий из бестриггерного потока данных с детектора CBM~RICH в среде CbmRoot.
\item Проведены пучковые тесты прототипа системы считывания и сбора данных в составе полнофункционального прототипа детектора CBM~RICH и дополнительные тесты на лабораторном стенде. 
\item Проведено комплексное исследование свойств канала считывания и сбора данных для CBM~RICH, реализованного на основе многоанодного ФЭУ H12700 с системой динодов ``metal channel'', специально разработанных передней электроники типа предусилитель-дискриминатор и высокоточного ВЦП с последующим прямым вводом данных в единую среду моделирования, сбора и анализа данных CbmRoot.
\item Исследованы временные свойства нанесенного на окно МА~ФЭУ сместителя спектра при возбуждении черенковскими фотонами.
\item Изучены возможности работы канала считывания при пониженных порогах.
\item Проведен сравнительный анализ особенностей считывания многоанодного ФЭУ временным и аналоговым трактами.
% \item Исследованы характеристики детектора CBM~RICH с учетом неидеальности геометрии и шумов электроники.
\end{enumerate}

\section*{На защиту выносятся следующие результаты:}

\begin{enumerate}
\item Разработка методологии и реализация ``CATIA-GDML geometry builder'', средства построения сложной, основанной на инженерном дизайне геометрии детекторов для моделирования прохождения и взаимодействия частиц.
\item Применение ``CATIA-GDML geometry builder'' для построения беспрецедентно точного параметризованного описания геометрии CBM~RICH в среде CbmRoot.
\item Реализация прототипа системы считывания и сбора данных CBM~RICH и проведение его тестов на пучке в составе полнофункционального прототипа этого детектора а также дополнительных тестов на лабораторном стенде.
\item Разработка алгоритмов и программного обеспечения для приема, упаковки и передачи бестриггерного потока данных, для калибровки точного времени и относительных задержек каналов и для построения событий из потока данных с детектора CBM RICH в среде CbmRoot.
\item Результаты комплексного исследования временных свойств канала считывания и сбора данных для CBM~RICH, реализованного на основе многоанодного ФЭУ H12700 с системой динодов ``metal channel'', специально разработанных передней электроники типа предусилитель-дискриминатор и высокоточного ВЦП с последующим прямым вводом данных в единую среду моделирования, сбора и анализа данных CbmRoot.
\item Исследование временных свойств нанесенного на окно МА ФЭУ сместителя спектра при возбуждении черенковскими фотонами.
\item Сравнительный анализ особенностей считывания многоанодного ФЭУ временным и аналоговым трактами.
% \item Исследование характеристик детектора CBM~RICH с учетом неидеальности геометрии и шумов электроники.
\end{enumerate}

\section*{Представление основных положений и результатов}

% Приведено в обратном хронологическом порядке. Можно перегруппировать по важности.

Основные положения и результаты работы докладывались и обсуждались на научных семинарах ЛИТ ОИЯИ и на различных международных конференциях и совещаниях, в том числе:

\begin{enumerate}

\item Семинар НОВФ ЛИТ ОИЯИ, Дубна, Россия, 22.12.2016 \\
Устный доклад ``Development of the readout and DAQ system for CBM RICH and EXPERT. 'CATIA-GDML geometry buidler' and Monte-Carlo geometry of CBM RICH.''

\item Семинар ``Contribution of the young Russian scientists into the project FAIR'', ИТЭФ, Москва, Россия, 14-15.12.2016 \\
Устный доклад ``Detailed study of the stability and uniformity of the CBM RICH readout and DAQ prototype characteristics. Development and application of the Monte Carlo geometry package''

\item Международная конференция ``The 9th International Workshop on Ring Imaging Cherenkov Detectors (RICH 2016)'', Блед, Словения, 05-09.09.2016 \\
Представлен постер ``Development of the CBM RICH readout electronics and DAQ''

\item Международная конференция ``The 20th IEEE-NPSS Real Time Conference (IEEE-NPSS RT2016)'', Падуя, Италия, 05-10.06.2016 \\
Представлен постер ``Development of the CBM RICH readout and DAQ''

\item Международная конференция ``The XX International Scientific Conference of Young Scientists and Specialists (AYSS-2016)'', ОИЯИ, Дубна, Россия, 14-18.03.2016 \\
Устный доклад ``Development and characterization of CBM RICH readout and DAQ''

\item Семинар ``Contribution of the young Russian scientists into the project FAIR'', ИТЭФ, Москва, Россия, 15-17.12.2015 \\
Устный доклад ``Development of 'CATIA-GDML geometry builder' and CBM RICH software''

\item Международное совещание ``26th CBM Collaboration Meeting'', Прага, Чехия, 14-18.09.2015 \\
Устный доклад ``PADIWA test measurements, beamtime analysis (TOT, WLS time resolution)''

\item Международное совещание ``25th CBM Collaboration Meeting'', ГСИ, Дармштадт, Германия, 20-24.04.2015 \\
Устный доклад ``Beamtime analysis: FLIB readout, TOT, timing''

\item Семинар ``Contribution of the young Russian scientists into the project FAIR'', ИТЭФ, Москва, Россия, 12-13.11.2013 \\
Устный доклад ``Modernization of simulation and data acquisition packages of CBM experiment''

\item Международная конференция ``20th International Conference on Computing in High Energy and Nuclear Physics (CHEP)'', Амстердам, Нидерланды, 14-18.10.2013 \\
Представлен постер ``Development and application of CATIA-GDML geometry builder''

\end{enumerate}

\section*{Публикации}


\section*{Личный вклад:}

\todo
\section*{Структура и содержание}\label{sec:secStructureAndContent}

Диссертация состоит из настоящего введения, пяти глав и заключения.

В первой главе
описываются условия эксплуатации, компоновка и основные свойства детекторов эксперимента CBM;
обсуждается важность точного описания и оптимизации конструкции детекторов в свете жёстких условий эксплуатации;
формулируется задача разработки инструментария для обмена геометрической информацией между САПР и средами Монте-Карло моделирования прохождения частиц через вещество (Geant4/ROOT) и облегчения создания детальных геометрических моделей для Geant4/ROOT;
детально описывается конструкция детектора CBM RICH и проводится сравнение с аналогичными приборами из ряда других экспериментов;
формулируются конкретные задачи, связанные с описанием геометрии детектора CBM RICH в среде Монте-Карло;
обсуждается концепция системы считывания и сбора данных эксперимента CBM и воплощение этой концепции в системе считывания и сбора данных детектора CBM RICH;
формулируется задача на исследование прототипа системы считывания и сбора данных указанного детектора.

Во второй главе
обсуждаются некоторые наиболее распространённые способы представления геометрических моделей в ЭВМ, используемые в различном ПО для решения различных вычислительных задач;
рассматриваются предпосылки и принципы создания инструментария, так называемого ``CATIA-GDML geometry builder'', для обмена геометрической информацией между САПР и средами Монте Карло моделирования прохождения частиц через вещество (Geant4/ROOT) и облегчения создания детальных геометрических моделей для Geant4/ROOT;
обсуждаются реализация отображения примитивов и иерархии объемов Geant4/ROOT на дерево построений в среде CATIA~v5 и набор созданных макропрограмм, входящих в ``CATIA-GDML geometry builder'';
описывается методика применения ``CATIA-GDML geometry builder'' и приводятся некоторые примеры.

Третья глава
посвящена применению пакета ``CATIA-GDML geometry builder'' для детектора CBM RICH. В ней подробно рассмотрены задачи, требующие точного описания геометрии механических конструкций детектора и системы крепления и позиционирования зеркал, учета эффектов, связанных с отклонением позиционирования зеркал от номинального, размещения и экранирования от магнитного поля фотодетекторов. Описаны созданные параметризованные геометрические модели и проведенные с их помощью исследования свойств детектора CBM RICH. Кроме того, даются рекомендации по эффективному применению использованного инструментария.

В четвёртой главе
описаны архитектура бестриггерной системы считывания и сбора данных CBM RICH, разработанные модули ПО, необходимые для сбора и анализа данных, а также экспериментальные установки, позволившие осуществить всестороннее исследование прототипа указанной системы.

Пятая глава
посвящена анализу данных пучковых и лабораторных тестов прототипа детектора CBM RICH и результатам исследования свойств и характеристик прототипа системы считывания и сбора данных. Здесь же, на основании проведенных исследований, даются рекомендации по модификации следующей версии прототипа системы считывания и сбора данных.

В заключении приводятся основные результаты работы и выражаются благодарности.

