\section{LeadingEdgeDiff}\label{sec:secLeadingEdgeDiff}

Один из этапов обработки данных --- построение событий. В данной работе рассматривается два типа событий --- сигналы от лазера и сигналы от черенковского кольца. В любом случае, событие --- это структура данных, содержащая информацию о хитах, сгруппированных по времени. Каждый хит содержит, как минимум, временную отметку момента прихода переднего фронта сигнала и номер канала, который в случае CBM~RICH указывает номер пикселя фоточувствительной камеры, т.е. говорит о геометрическом положений зарегистрированного фотона.

Данное исследование посвящено, в первую очередь, временным характеристикам системы считывания, поэтому в основном речь пойдёт о временных отметках.

Очевидно, что для каждого события можно построить несколько распределений, которые на большом массиве данных, т.е. на многих событиях, характеризуют систему считывания и могут быть использованы для калибровки электроники с целью повышения временного разрешения системы. Т.к. событие имеет максимальную ширину, определяемую размерами окна в алгоритме построения событий, распределения могут иметь ``обрезанные хвосты'', которые, однако, невозможно избежать.

Пусть событие содержит N хитов. Введём внутри события нумерацию хитов от 0 до N. Пусть внутри события хиты упорядочены по времени, т.е. хит с временной отметкой $ t_{0} $ был зарегистрирован раньше остальных, а хит с временной отметкой $ t_{N} $ --- позже всех. Такой порядок может, например, обеспечиваться естественным образом алгоритмом построения событий. Внутри события все временные отметки зарегистрированы в разных каналах --- множественные хиты в одном канале в одном событий являются признаком того, что порог дискриминатора установлен слишком низко и регистрируются шумы. Введём в рассмотрение распределение $ \omega $ разностей временных отметок всех хитов, кроме первого, относительно первого, т.е. распределение

{\centering
$ t_{j} - t_{0} $, где  $ j \in [1..N] $.\\
}

Также введём распределение $ \sigma_{1} $ всех пар временных отметок одного события, т.е.

{\centering
$ t_{j}-t_{i} $, где $ i \in [0..N], j \in [0..N], i \neq j $.\\
}

Очевидно, что в такой формулировке одна и та же пара временных отметок войдёт в распределение дважды с разными знаками --- например, $ t_{1}-t_{2} $ и $ t_{2}-t_{1} = -(t_{1}-t_{2}) $. Это делает распределение симметричным, среднее значение строго равно 0, а ширина распределения чуть больше, чем в случае, когда нет дублирования информации. Введём непрерывную нумерацию каналов и примем, что в разности $ t_{j}-t_{i} $ первая временная отметка была зарегистрирована каналом $a$, а вторая --- каналом $b$. Введём распределение $ \sigma_{2} $, по сути очень похожее на $ \sigma_{1} $, но без дублирования информации, в котором будем учитывать только пары, у которых $ b > a $.

В идеальной ситуации, если событие соответствует одной вспышке лазера или одному черенковскому кольцу, и отсутствуют факторы, размывающие время регистрации, все разницы были бы равны нулю. В качестве таких размывающих факторов можно привести, например, следующие: временные характеристики лазера, разброс геометрических путей черенковских фотонов, разброс времени прохождения электронной лавины в динодной системе ФЭУ, дребезг сигналов в передней электронике. Из-за перечисленных явлений распределение $ \omega $ имеет следующую форму --- (описание). Распределение $ \sigma_{2} $ --- (описание).

%Распределение $ \sigma_{2} $ позволяет определить относительную задержку распространения сигнала,
%Влияние всех этих явлений выливается в то, что распределение $ \sigma $ имеет колоколообразную форму.

Среднее значение либо положение масимума распределения $ \sigma_{2} $ можно использовать для того, чтобы определить значение поправки для данной пары каналов. Если выполнить анализ с применением коррекций, то вид всех распределений изменится. $ \omega $ сгруппируется ближе к нулю, $ \sigma_{2} $ передвинется к нулю, а $ \sigma_{1} $ сузится к нулю.

Представляется возможность анализировать различные области фоточувствительной камеры. Интересно группировать хиты в соответствии с тем, какой электроникой они обрабатываются. В данном анализе было введено 4 подмножества: 1~пара каналов, 16~каналов одной платы передней электроники, 64~канала одного МА~ФЭУ, 256~каналов 4~МА~ФЭУ, образующих площадку 2х2~МА~ФЭУ в одном углу камеры. При том, что вся фоточувствительная камера на пучковых тестах имела размер 4х4~МА~ФЭУ, рассматривать более 4~МА~ФЭУ одновременно не имеет смысла, т.к. в прототипе были установлены различные модели МА~ФЭУ, некоторые покрытые сместителем спектра, а некоторые нет.
