КГП --- Кварк-Глюонная Плазма
КХД --- Квантовая ХромоДинамика
FAIR --- Facility for Antiproton and Ion Research, центр по исследованию протонов и антипротонов
SIS --- SchwerIonenSynchrotron, синхротрон тяжёлых ионов
CBM --- Compressed Baryonic Matter, сжатая барионная материя
TDR --- Technical Design Report
MVD --- Micro Vertex Detector, вершинный микродетектор
STS --- Silicon Tracking System, кремниевая трековая система
RICH --- Ring Imaging CHerenkov detector, детектор черенковских колец
MUCH --- MUon CHambers, мюонные камеры
TRD --- Transition Radiation Detector, детектор переходного излучения
TOF --- Time Of Flight detector, времяпролётный детектор
ECAL --- Electromagnetic CALorimeter, электромагнитный детектор
PSD --- Projectile Spectator Detector
RHIC --- the Relativistic Heavy Ion Collider, релятивистский коллайдер тяжёлых ионов
NICA --- Nuclotron-based Ion Collider fAciliy, коллайдерный комплекс на базе нуклотрона
LHC --- Large Hadron Collider, большой адронный коллайдер
LHCb --- LHC beauty experiment
ALICE --- A Large Ion Collider Experiment
MPD --- Multi-Purpose Detector
HERA-b --- Hadron-Electron Ring Accelerator beauty
DAQ --- Data AcQuisition, сбор данных
FLES --- First Level Event Selector, система отбора событий первого уровня
FLIB --- FLES Interface Board, плата интерфейса системы отбора первого уровня
CMOS --- Complementary Metal-Oxide-Semiconductor, комплементарная структура металл-оксид-полупроводник (КМОП)
MAPS --- Monolithic Active Pixel Sensor
FPGA --- Field-Programmable Gate Array, программируемая пользователем вентильная матрица (ППВМ)
GEANT --- GEometry ANd Tracking
МК --- Монте-Карло
CATIA --- Computer Aided Three-dimensional Interactive Application
САПР --- Система Автоматизированного ПРоектирования
GDML --- Geometry Description Markup Language

МА~ФЭУ --- МультиАнодный (\todo МультиАнодный) Фотоэлектронный Умножитель
FEB --- Front-End Board
ROC --- ReadOut Controller
DCB --- Data Combuner Board

