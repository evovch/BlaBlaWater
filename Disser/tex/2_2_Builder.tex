\section{``CATIA-GDML geometry builder''}\label{sec:Builder}

``CATIA-GDML geometry builder'' (далее просто ``Builder'') представляет собой набор документов-шаблонов и макропрограмм для САПР CATIA~v5 вместе с настройками окружения и инструкциями к применению стандартных средств CATIA~v5. ``Builder'' ставит своей задачей упростить процесс создания CSG моделей с иерархией объёмов, напрямую совместимых с GEANT/ROOT.

Центральная идея ``Builder'' заключается в правилах соответствия сущностей CATIA~v5 и сущностей геометрии в GEANT/ROOT. Это соответствие делает возможным конвертацию MC-модели в CATIA~v5 в любой внешний файл с целью дальнейшего импорта в ROOT/GEANT. В качестве формата для обмена был выбран XML-подобный формат GDML (geometry description markup language), разработанный в CERN, для которого в GEANT4 и ROOT реализованы методы импорта и экспорта.

Вся геометрия установки создаётся в одном документа типа CATProduct. Объёму соответствует деталь, хранящаяся в файле типа CATPart. Форме соответствует главное тело детали, по умолчанию называемое PartBody. В CATIA не записывается описание материала так, как это принято в GEANT/ROOT, а сохраняется только имя материала в пользовательском параметре Material. Это возможно по той причине, что существует практика хранить описание материалов во внешнем файле или базе данных, считывать его перед выполнением моделирования и приписывать объёмам в соответствии с именами. Для обозначения физического объёма $B$ внутри $A$ в структуре документа, описывающего объём $A$, создаются тела Body.A.*, где * по умолчанию обозначает номер вхождения, но допускается запись любой идентифицирующей строки.

\textbf{Может быть частично перенести в описание методов описания геометрии в GEANT/ROOT.}
Также в ``Builder'' предусмотрена возможность задания геометрии некоторыми продвинутыми методами, специфичными для GEANT/ROOT. В GEANT/ROOT существует тип объёvов, называемый Asssembly, который характеризуется тем, что он не имеет формы и материала. Практически объём типа Assembly является контейнером без границ, который объединяет свои дочерние объёмы, что особенно удобно как минимум в двух случаях. Во-первых, если необходимо многократно позиционировать группу объёмов, которую невозможно охватить простой формой. Во-вторых, если преобразование координат при позиционировании одного или группы объёмов имеет сложную структуру и удобно представить его как суперпозицию двух преобразований. Как частный случай можно упомянуть ситауцию, когда какой-либо параметр преобразования является параметром модели (см. секцию~\ref{sec:Parameterization})

Один из плюсов ``Builder'' заключается в том, что пользователю предоставляется возможность работать с полноценной инженерной моделью и MC-моделью в одной и той же среде, имеющей широкие возможности для анализа и редактирования геометрии. ``Builder'' не ставит своей задачей перевод модели из одного геометрического представления в другой, но значительно ускоряет процесс создания одной геометрии, на основе другой. Также важно отметить, что подходы к геометрическому моделированию в САПР подразумевают широкое использование параметров --- практически все размеры, значения поворотов и сдвигов, количество вхождений в массивы и прочие числа, определяющие форму и структуру модели, могут подвергаться изменению на любом этапе. Если в процесе построения модели этот принцип параметризированного моделирования нарушается, САПР предупреждает пользователя перед выполнением операции, которая приведёт к разрыву связи с параметром. В CATIA~v5 при изменении каких-либо параметров геометрическая модель перестраивается интерактивно --- от долей секунды до нескольких секунд в зависимости от сложности модели.  Эта стандартная черта САПР очень удобна при работе с MC-моделями и отсутствует, например, в GEANT и ROOT.

``Builder'' включает в себя файлы, в которых специальным образом построены примитивы GEANT/ROOT, позволяющие пользователю при построении MC-модели не вникать в подробности реализации, а использовать их практически как и в процессе создания геометрии средствами ROOT или GEANT. ``Builder'' также включает в себя макропрограммы для CATIA~v5, которые также ставят своей задачей сделать процесс построения геометрии в ``Builder'' максимально похожим на процесс построения геометрии в GEANT или ROOT. Основной макрос --- это конвертер ``CATIA2GDML'', который проецирует дерево построения модели в CATIA в GDML файл. Также разработан обратный конвертер ``GDML2CATIA'' для импорта GDML файлов.

Целевая аудитория ``CATIA-GDML geometry builder'' --- физики, владеющие CATIA~v5 на базовом уровне, и инженеры, продвинутые пользователи САПР, изучившие способ представления геометии в GEANT/ROOT хотя бы на теоретическом уровне.

\textbf{Есть опыт, который показывает, что для достижения такого уровня как физикам, так и инженерам, достаточно прохождения двухнедельного курса.}

Предлагается новый алгоритм работы, в котором ``Builder'' используется как многофункциональный инструмент. Описанный ниже алгоритм сформулирован на успешном опыте разработки CBM RICH на протяжении 3 (4) лет.

Задача создания и поддержания актуальной MC-модели поручается ответственному человеку, владеющему CATIA, GEANT/ROOT и ``CATIA-GDML geometry builder''.

В зависимости от того, какая информация и в каких файлах имеется к началу работы, алгоритм немного различается.
Если разработка ведётся в нуля и нет никаких данных в ЭВМ, что возможно, например, когда проект находится на таком этапе, когда нужно выполнить грубое моделирование, показывающее принципиальную возможность реализации, то наиболее оптимальный способ --- сразу строить MC-модель в CATIA средствами ``Builder''.
Если, скажем, проект находится на раннем этапе разработки и уже имеется какая-то приблизительная САПР модель, то рекомендуется импортировать её стандартными средствами CATIA, чтобы затем на её основе построить MC-модель в CATIA в автоматизированном режиме с помощью средств ``Builder''. Инженерную геометрию можно импортировать практически из любой САПР, например с помощью широко распространённого формата STEP. 
Третий распространённый случай это когда уже имеется некоторая MC-модель в конечной системе моделирования. Как в GEANT4, так и в ROOT имеется стандартная возможность экспортировать геометрию в GDML файл без потери информации. Эту возможность могут наследовать все дочерние пакеты (как FairRoot и далее CbmRoot), но для этого необходимо явно активировать функциональность GDML. В этом случае можно импортировать модель в CATIA в MC-формате, однако иногда требуются некоторые дополнительные ручные операции после импорта. Они выполняются однократно и лишь делают структуру документа более оптимальной, но не изменяют геометрию.

Во всех этих алгоритмах, независимо от типа и количества исходных данных, получаются файлы CATIA в формате ``Builder'', которые в дальнейшем будут являться основными (первичными) файлами для получения рабочей MC-модели в экспериментальном пакете, которым в случае CBM RICH является CbmRoot. Модель из CATIA экспортируется в GDML файл, который не требует каких-либо последующих изменений в структуре. Для достижения этого условия была проведена огромная работа по мере разработки MC-модели CBM RICH. Допускается и даже рекомендуется текстовое редактирование GDML файла, но только для изменения значений параметров в define секции у параметризованных моделей. Затем, по желанию коллаборации, GDML файл может быть конвертирован в бинарный ROOT-файл, который содержит геометрию, которую невозможно редактировать. Это защищает модель от случайных изменений, что особенно актуально в случае параметризованных моделей. Соответственно, если требуется изменить значения параметров, пользователь может отредактировать GDML файл и экспортировать в новый ROOT файл. Практика показывает, что в случае, если требуется множество файлов с MC-геометрией, то обязательно нужно писать комментарии --- либо в самом GDML файле, либо в текстовом файле рядом с GDML/ROOT файлом. Обычно в коллаборации вводят правила именования файлов.

\subsection{Примитивы в ``Builder''}\label{sec:Primitives}

%В принципе мотивация уже обсуждалась в секции выше.
Примитив можно построить стандартными средствами САПР, используя эскизы и формообразования, но в таком случае конвертер не сможет  автоматически определить, является ли построенная форма примитивом и, если да, определить параметры примитива. По этой причине был разработан принцип хранения формы примитива в MC-модели с помощью средства CATIA~v5, называемого User-Defined Feature (UDF), и средства для автоматизации создания примитивов --- макросы \macroname{AddShape} и \macroname{Poly}. Каждый примитив реализован в своём файле типа CATPart, в котором создаётся описание UDF, превращая этот файл в шаблон. Некоторые объекты модели, в случае примитивов --- некоторые стандартные плоскости, и параметры модели объявляются <<внешними>>. Далее в другом документе возможно создать вхождение формы, определённой в файле-шаблоне, вызвав соответствующее формообразование. При этом в текущем документе потребуется лишь выбрать необходимые элементы, с которыми будут совпадать <<внешние>> объекты шаблона, и задать значения параметрам создаваемого вхождения.

\subsection{Макропрограммы для CATIA~v5}\label{sec:Macros}

Макропрограммы для CATIA~v5 написаны на VBA с применением CATIA API. Все макропрограммы, кроме \macroname{AddShape} и \macroname{Poly}, доступны пользователю в режиме работы над сборкой. В CATIA различают открытый документ (верхний в дереве в текущем окне), активный документ (синий), выделенный объект (оранжевый) и рабочий объект (подчёркнутый). Пользователь может выполнить все операции, необходимые для получения MC-модели, самостоятельно без применения макропрограмм, но в этом случае велика вероятность упустить какой-либо шаг, что приведёт к ошибке, которую сложно диагностировать.

В MC-модели в CATIA есть строгие правила именования. Применение макросов избавляет пользователя от необходимости контролировать имена объектов в документах. Все имена, сгенерированные при использовании ``Buider'' не конфликтуют между собой и позволяют получить корректный GDML файл на выходе. Практически везде пользователь имеет право изменять суффиксы, не изменяя основного названия, несущего информацию о типе объекта --- формообразования, тела, и т.д. Однако в редких случаях суффикс имеет решающее значение, как например в именах поворотах (напр, ``Rotate.X'') суффикс несёт информацию о оси поворота.

В процессе разработки ``Builder'' был выработан стандартный алгоритм создания геометрии. Первый шаг --- создание нового документа типа CATProduct, который в дальнейшем будет единственным продуктом, и его сохранение на диск. Этот продукт будет представлять модель всей экспериментальной установки. Второй этап --- наполнение продукта описанием объёмов без описания взаимосвязей между ними. Для этого используется макрос \macroname{AddNewPart}, который автоматически открывает в отдельном окне новый документ типа CATPart, сформированный из специального шаблона и соответствующий создаваемому объёму. Система переходит в режим редактирования детали, где доступны только два макроса \macroname{AddShape} и \macroname{Poly} для создания формы объёма. Здесь же можно и задать имя материала объёма. По окончании редактирования нового объёма в отдельном окне пользователь должен сохранить активный документ и закрыть это окно. CATIA при этом возвращается к редактированию продукта. После того, как созданы объёмы, заданы формы и, возможно, имена материалов, алгоритм подразумевает задание иерархии объёмов, то есть позиционирование одних объёмов в других. Для этого в ``Builder'' существует целый ряд макропрограмм для создания различных типов взаимосвязей --- \macroname{Inserter}, \macroname{ArrayMaker}, \macroname{Replica}. После того, как выполнено размещение дочернего объёма $A$ в материнском объёме $B$, пользователь может указать поворот и сдвиг, задающие матрицу позиционирования $A$ в $B$. Для упрощения расчётов в некоторых случаях очень удобно применять макропрограммы \macroname{PointToPointAligner} (\macroname{Pt2PtAligner}), \macroname{Mover} и \macroname{Measure}. Для удобного редактирования материалов всех объёмов был разработан менеджер материалов \macroname{MaterialsManager}, который обычно имеет смысл вызывать перед экспортом для проверки ранее заданных имён материалов, либо назначения новых. Также перед экспортом рекомендуется проверить модель на наличие ошибок с помощью макроса \macroname{Checker}. В конце выполняется экспорт макросом \macroname{CATIA2GDML}. Отдельно стоят макропрограммы \macroname{Duplicator} для создания множественных идентичных, но не связанных, параметризованных подсборок и обратный конвертер \macroname{GDML2CATIA} для импорта GDML файла.

Для комфортной работы с ``Builder'' в поставке также имеются файлы для настройки окружения CATIA. Использования окружения в принципе не обязательно, но часть функционала зависит от путей к файлам, которые прописаны в переменных окружения, поэтому настоятельно рекомендуется перед использованием ``Builder'' выполнить настройку, следуя инструкции, поставляемой в пакете.

\subsubsection{AddNewPart}\label{sec:AddNewPart}

Данная макропрограмма автоматизирует создание нового документа типа CATPart на основе шаблона, содержащего необходимые элементы --- публикация главного тела детали, называемая PartBody, пользовательский параметр под названием Material со значением по умолчанию ???. Также для удобства погашены стандартные плоскости.

\subsubsection{AddShape}\label{sec:AddShape}

\macroname{AddShape} используется для создания примитивов, в случае необходимости вместе с поворотами и сдвигом. Макропрограмма играет роль интерфейса между пользователем и файлами примитивов. При запуске макроса выводится окно со списком доступных примитивов, по нажатии на кнопку ``создать'' в рабочее тело детали вставляется выбранный примитив со значениями параметров по умолчанию. Если на форме графического интерфейса выбраны флаги создания поворотов и сдвига, то создаются соответствующие формообразования.

\subsubsection{Poly}\label{sec:Poly}

В силу ограничений CATIA нет возможности представить полипримитивы (polycone и polyhedra) с помощью тех же средств, что и остальные примитивы, поэтому для них была разработана специальная структура дерева и правила именования. Для автоматизации построения полипримитивов в соответствии с этой структурой предоставляется макрос \macroname{Poly}. Секции поликонуса представлены стандартными конусами. В случае polyhedra для представления секции используется hedra --- специальный примитив, не поддерживаемый GEANT/ROOT.

\subsubsection{Inserter}\label{sec:Inserter}

Макрос \macroname{Inserter} --- это инструмент для помещения одного выбранного объёма в другой. Также можно сказать, что \macroname{Inserter} создаёт физический объём, задающий связь материнский-дочерний между двумя существующими логическими объёмами. \macroname{Inserter} --- возможно, самый используемый макрос, в результате работы которого в документе типа CATPart, представляющем материнский объём, создётся тело с именем ``Body.B.*'', где $B$ --- имя дочеhнего объёма. Внутри этого тела имеется ссылка на публикацию PartBody документа типа CATPart, представляющего объём $B$, и элементы преобразования типа Rotate и Translate --- три поворота и сдвиг, задающие матрицу позиционирования $B$ внутри $A$.

\subsubsection{ArrayMaker}\label{sec:ArrayMaker}



Макрос \macroname{ArrayMaker} схож с \macroname{Inserter} по идее и реализации. После выполнения 

\subsubsection{Replica}\label{sec:Replica}

\subsubsection{PointToPointAligner}\label{sec:PointToPointAligner}

\subsubsection{Mover}\label{sec:Mover}

\subsubsection{Measure}\label{sec:Measure}

\subsubsection{MaterialsManager}\label{sec:MaterialsManager}

Приложение \macroname{MaterialsManager} предоставляет пользователю возможность изменять материалы отдельных объёмов, находясь на уровне виртуальной сборки. Это избавляет от необходимости часто переключаться между документами либо режимами работы CATIA при контекстном редактировании. Также заметным преимуществом использования \macroname{MaterialsManager} является наглядность --- информация о материалах всех объемов представляется в компактном списке, присутствует возможность быстро изменять значения в нескольких элементах списка. Помимо этого, наличие \macroname{MaterialsManager} позволяет отложить работу с материалами на последний этап. Использование шаблона файла детали предотвращает от того, что пользователь вообще не укажет материал объема --- по умолчанию указан вакуум.

\subsubsection{Checker}\label{sec:Checker}

Существует необходимость проверять правильность построенной пользователем MC-модели в CATIA перед тем как выполнять экспорт в GDML. Она возникает в силу того, что в разработанной структуре документов CATIA для MC-модели введено множество правил и ограничений, нетипичных для conventional использования системы. \macroname{Checker} выполняет 2 типа проверок. Первый --- корректность с точки зрения конвертера, т.е. соблюдение структуры документов, правильность именования, второй --- корректность с точки зрения правил построения геометрии в GEANT/ROOT.

Использование разработанных интерактивных приложений ограждает пользователя от ошибок именования в итоговой MC-модели. Есть только одно место, где необходимо вручную указывать имя --- формообразование-вращение при позиционировании операнда на уровне формы (?уточнить?).

Корректность геометрии определяется по двум критериям:
\begin{enumerate}
\item любые два объёма, находящиеся на одном уровне, не должны пересекаться;
\item любой дочерний объём не должен выходить за пределы материнского объёма.
\end{enumerate}

Чтобы организовать проверку указанных условий, разработан специальный алгоритм и реализован в виде отдельной макропрограммы \macroname{Checker} САПР CATIA. В цикле перебираются все пары объёмов, лежащих на одном уровне, и проверяется, не пусто ли множество пересечения текущей пары. В том случае, если не пусто, то возникает событие, оповещающее, что текущая пара объёмов расположена недопустимым образом.

Более детально описанный процесс выглядит следующим образом. В силу того, что физические объёмы описываются телами детали, перебор объёмов, лежащих на одном уровне, сводится к перебору всех тел детали, кроме PartBody. Чтобы исследовать множество пересечения пары тел, создаётся новое пустое тело и копии исходных. Затем применяется булева операция Intersect над копиями, и результат заносится в ранее созданное пустое тело. На этапе выполнения формообразования булевой операции выдаётся возможность отследить корректность результата. С точки зрения CATIA пустое пересечение является ошибочным результатом и возникает внутренняя ошибка. Именно программный отлов и обработка этой ошибки говорит о корректности расположения объёмов. После выполнения булевой операции результирующее тело удаляется, не оставляя таким образом никаких следов промежуточных преобразований.

Для того чтобы отследить, не выходит ли объём за пределы материнского объёма, применяется схожий подход. Отличие заключается в последовательности булевых операций --- вместо пересечения $A*B$ двух объёмов $A$ и $B$ проверяется объём, полученный последовательностью двух операций $(A+B)-A$, где $A$ --- материнский объём, а $B$ --- дочерний. Присутствие результата операции $(A+B)-A$ говорит о том, что какая-либо часть дочернего объёма расположена за пределами материнского.

\subsubsection{CATIA2GDML}\label{sec:CATIA2GDML}


\subsubsection{GDML2CATIA}\label{sec:GDML2CATIA}

\macroname{GDML2CATIA} выполняет процедуру, обратную \macroname{CATIA2GDML} --- проецирует GDML файл на дерево модели CATIA~v5. В ``Builder'' есть возможность задавать линейные и круговые массивы --- многократные вхождения дочернего объёма в материнский, позиционированные с некоторым шагом вдоль линейной или круговой оси соответственно. В MC-модели в CATIA для массивов применяется соответствующее стандартное формообразование pattern. Такая возможность отсутствует в GDML, поэтому при экспорте из CATIA в GDML выполняется расчёт поворотов и сдвигов для каждого элемента массива и они представляются как отдельные, независимые дочерние объёмы. Таким образом при конвертации в обратном направлении, из GDML в CATIA, невозможно восстановить массив. Следовательно, одна из немногих (единственная?) операций, которые необходимо совершать после импорта геометрии из GDML в CATIA --- ручной перевод множества дочерних объёмов в массив. Обычно это очень простая процедура, и заключается она в том, что удаляются все вхождения, кроме первого, и в список формообразований первого тела добаляется pattern, которому задаются необходимые параметры и имя.

\subsubsection{Параметризация}\label{sec:Parameterization}

Одна из наиболее важных возможностей ``CATIA-GDML geometry builder'' --- это возможность создания параметризованных геометрических MC-моделей. У параметризованной модели имеются входные параметры и формулы, задающие зависимость между этими входными параметрами и внутренними переменными, такими как параметры примитивов, значения поворотов и смещений. Данная концепция хорошо ложится на методы работы с геометрией в САПР, особенно CATIA~v5. Также параметризация поддерживается форматом GDML и импортёрами GEANT4(?) и ROOT.

В модели CATIA~v5 можно вводить пользовательские параметры как в документах типа CATProduct, так и в документах типа CATPart. Причём сборка в CATProduct файле может иметь свои пользовательские параметры и формулы, а дочерние компоненты в CATPart файлах --- свои. Обязательное требование ``Builder'' таково, что все параметры и формулы должны находиться в верхнем продукте. CATIA~v5 позволяет задавать зависимости между любыми параметрами, в том числе внутренними, не являющимися пользовательскими, однако для успешного экспорта в GDML файл формула должна в левой части иметь параметр примитива или угол поворота или значение сдвига, а в правой части --- формулу только над пользовательскими параметрами. Пользовательский параметр CATIA~v5, экспортируемый в переменную в GDML должен обязательно иметь безразмерный тип Real. В связи с этим имеются правила оформления формул и приведения единиц измерения. Также имеется стандартная переменная DEGtoRAD для перевода значения углов из градусов в радианы.

На выходе получается GDML файл, у которого в define секции есть тэги variable, обозначающие входные параметры модели со значениями. При импорте параметризованной геометрии из GDML в ROOT все значения внутренних переменных рассчитываются в соответствии с формулами по значениям входных параметров и параметризация теряется. Следовательно значения входных параметров должны задаваться пользователем непосредственно в GDML файле перед импортом в конечную систему.

\subsubsection{Duplicator}\label{sec:Duplicator}

Создание MC-модели более-менее сложной экспериментальной установки обычно требует создания нескольких вхождений параметризованных подборок с разными значениями параметров. Можно привести следующий пример. Рассмотрим детектор, состоящий из однотипных модулей, содержащих массив чувствительных объёмов (сенсоров), и какие-то другие элементы, например, платы передней электроники. Предположим, что существует несколько типоразмеров модулей, отличающихся количеством и размером сенсоров. Таким детектором может быть, например, калориметр, построенный из нескольких типов модулей, отличающихся гранулярностью --- размер чувствительного объёма увеличивается по мере удаления от пучка. Очевидно, что если типы модулей отличаются лишь значениями каких-либо переменных, то представляется возможным построить одну параметризованную модель модуля, чтобы дальше использовать её многократно для построения всего детектора. К сожалению в GEANT/ROOT нет возможности так сделать --- нужно для каждого типа иметь отдельное определение геометрии. Также это невозможно и в GDML и в CATIA.
Для того, чтобы 


\subsection{Избранные подробности реализации ``CATIA-GDML geometry builder''}

Каждый макрос ``Builder'' --- это VBA проект, который хранится в отдельном catvba файле. Проект состоит из трёх разделов --- элементы графического интерфейса (формы), модули и модули классов. Большинство макросов ``Builder'' написано в соответствии с идеологией структурного программирования, без применения классов, и разделение на модули выполнено из соображений читаемости кода. Обычно в отдельный модуль выносился функционал, объединённый некоторой задачей. Так, например, во многих макросах имеется модуль ??? (имя) для продвинутой работы со строками, модуль ??? (имя) для ??? (задача). В некоторых случаях естественным образом требовалось использовать классы. Так, например, был реализован класс матрицы с методами нахождения углов поворота, который использовался в ??? и более подробно описан в~\ref{sec:Matrices}.

\subsubsection{Работа с матрицами позиционирования в ``CATIA-GDML geometry builder''}\label{sec:Matrices}
