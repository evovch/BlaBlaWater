\section*{Научная новизна результатов, полученных автором / Новизна и практическая ценность:}

\begin{enumerate}
\item Разработана схема отображения иерархии геометрии, используемой в моделировании транспорта частиц методом Монте Карло (МК), на дерево построений САПР CATIA~v5.
\item В среде CATIA создан набор шаблонов для примитивов и сущностей конструктивной твердотельной геометрии, принятой в системах МК моделирования детекторов.
\item Создан набор инструментов для полуавтоматического построения детальной МК геометрии на основе САПР модели и быстрого обмена геометрией между САПР CATIA~v5 и пакетами МК моделирования GEANT и ROOT.
\item Выполнены беспрецедентно точные параметризованные описания ряда приборов и детекторов в средах МК моделирования.
\item На основе детального параметризованного описания геометрии CBM~RICH выполнена оптимизация компоновки детектора. 
\item Собран прототип системы считывания и сбора данных детектора CBM~RICH.
\item Разработано программное обеспечение для приема, упаковки и передачи бестриггерного потока данных с прототипа системы считывания и сбора данных с частотой до 20~МГц.
\item Разработано программное обеспечение для калибровки точного времени и относительных задержек каналов в потоке данных с детектора CBM RICH.
\item Разработано программное обеспечение для построения событий из бестриггерного потока данных с детектора CBM~RICH в среде CbmRoot.
\item Проведены пучковые тесты прототипа системы считывания и сбора данных в составе полнофункционального прототипа детектора CBM~RICH и дополнительные тесты на лабораторном стенде. 
\item Проведено комплексное исследование свойств канала считывания и сбора данных для CBM~RICH, реализованного на основе многоанодного ФЭУ H12700 с системой динодов ``metal channel'', специально разработанных передней электроники типа предусилитель-дискриминатор и высокоточного ВЦП с последующим прямым вводом данных в единую среду моделирования, сбора и анализа данных CbmRoot.
\item Исследованы временные свойства нанесенного на окно МА~ФЭУ сместителя спектра при возбуждении черенковскими фотонами.
\item Изучены возможности работы канала считывания при пониженных порогах.
\item Проведен сравнительный анализ особенностей считывания многоанодного ФЭУ временным и аналоговым трактами.
\item Исследованы характеристики детектора CBM~RICH с учетом неидеальности геометрии и шумов электроники.
\end{enumerate}
