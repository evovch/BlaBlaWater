% http://www.slac.stanford.edu/econf/C060717/papers/L010.PDF

%Эксперимент ALICE (A Large Ion Collider Experiment) --- это один из четырёх крупных экспериментов на большом адронном коллайдере (Large Hadron Collider, LHC), который занимается изучением столкновений тяжёлых ионов.

%The rate of Pb--Pb collisions in 2010 and 2011 was well below the ALICE limits and ALICE was able to take data at the highest achievable luminosity, on the order of $10^25$ $s^{-1}$ $cm^{-2}$ in 2010 and $10^{26}$ $s^{-1}$ $cm^{-2}$ in 2011, with the corresponding hadronic $\mu$ being on the order of $10^{-5}$ -- $10^{-4}$ and $10^{-4}$ -- $10^{-3}$, respectively.

%During the 2011 Pb--Pb running period, the interaction rate provided by the LHC reached 3-4 kHz. ALICE ran with the minimum bias, centrality, and rare triggers activated at the same time. In the LHC Run 2 (2015--2017), for which the expected collision rate is O(10) kHz, still low enough to avoid pileup.

%The LHC at CERN will provide colliding Pb ions with an energy of $\sqrt{s_{NN}}=5.5 TeV$.

%It is expected that the LHC can deliver luminosities of $10^{27}$ $cm^{2}$ $s^{-1}$ for Pb--Pb collisions, which results in a minimum-bias interaction rate of 8~kHz. Lighter ions can be delivered with higher luminosities of up to $10^{29}$ $cm^{2}$ $s^{-1}$, corresponding to an interaction rate of several 100~kHz. The machine can deliver p--p luminosities up to $10^{31}$ $cm^{2}$ $s^{-1}$ but because of detector limitations this luminosity is restricted to $10^{30}$ $cm^{2}$ $s^{-1}$ for ALICE.

%--- Hard processes become abundant.
%The abundance of hard processes at the LHC will allow for precision test of perturbative QCD. In addition, the large jet rates at the LHC permit detailed measurements of jet quenching to study the early stages of the collision.

%--- Access to weakly interacting hard probes.
%Direct photons as well as $Z^{0}$ and $W^{\pm}$ bosons produced in hard processes will provide information about nuclear parton distributions at high $Q^{2}$ . Jet tagging with such probes yields a calibrated energy scale for jet quenching studies.

%--- Fireball expansion is dominated by parton dynamics.
%Due to the expected longer lifetime of the QGP, the parton dynamics will dominate over the hadronic contribution to the fireball expansion and the collective features of the event.

%The charged particle multiplicity per colliding nucleon pair measured by ALICE for the most central collisions is double that measured at RHIC, where the collision energy is factor 14 lower, fig.1. This shows that the system created at LHC has much higher energy density and is at least 30\% hotter that at RHIC. Fig. 2 shows the charged particle multiplicity as a function of the number of participants. 

%One of the classic signals expected for a quark-gluon plasma (QGP) is the radiation of ``thermal photons'', with a spectrum reflecting the temperature of the system. With a mean-free path much larger than nuclear scales, these photons leave the reaction zone created in a nucleus–nucleus collision unscathed. So, unlike hadrons, they provide a direct means to examine the early hot phase of the collision. However, thermal photons are produced throughout the entire evolution of the reaction and also after the transition of the QGP to a hot gas of hadrons. In the PbPb collisions at the LHC, thermal photons are expected to be a significant source of photons at low energies (transverse momenta, $p_{T}$, less than around 5~\GeVoverC{}). The experimental challenge in detecting them comes from the huge background of photons from hadron decays, predominantly from the two-photon decays of neutral pions and ? mesons. 

%Direct photons are defined as photons not coming from decays of hadrons, so photons from initial hard parton-scatterings (prompt photons and photons produced in the fragmentation of jets) --- i.e. processes already present in proton-proton collisions --- contribute to the signal. Indeed, for pT greater than around 4~\GeVoverC{}, the measured spectrum agrees with that for photons from initial hard scattering obtained in a next-to-leading-order perturbative QCD calculation. For lower pT, however, the spectrum has an exponential shape and lies significantly above the expectation for hard scattering, as the figure shows. The inverse slope parameter measured by ALICE, $T_{LHC}$ = 304$\pm$51 (stat.+syst.) MeV, is larger than the value observed in gold-gold collisions at $\sqrt{s}$ = 0.2~TeV at Brookhaven’s Relativistic Heavy-Ion Collider (RHIC), $T_{RHIC}$ = 221 $\pm 19$ (stat.) $\pm 19$ (syst.) MeV. In typical hydrodynamic models, this parameter corresponds to an effective temperature averaged over the time evolution of the reaction. The measured values suggest initial temperatures well above the critical temperature of 150--160 MeV (approx. 1.8 $\times$ 1012 K) at which the transition between ordinary hadronic matter and the QGP occurs. The ALICE measurement also indicates that the LHC has produced the hottest piece of matter ever formed in a laboratory. 

%At the LHC, however, extremely interesting developments are expected. In particular, a much higher number of charm-anticharm pairs are produced in the nuclear interaction, thanks to the unprecedented centre-of-mass energies. As a consequence, even a suppression of the $J/\psi$ yield in the hot QGP phase could be more than counter-balanced by a statistical combination of charm-anticharm pairs happening when the system, after expansion and cooling, finally crosses the temperature boundary between the QGP and a hot gas of particles. If the density of heavy quark pairs is large enough, this regeneration process may even lead to an enhancement of the $J/\psi$ yield --- or at least to a much weaker suppression with respect to the experiments at lower energies. The observation of the fate of the $J/\psi$ in nuclear collisions at the LHC constitutes one of the goals of the ALICE experiment and was among its main priorities during the first run of the LHC with lead beams in November/December 2010.

%The results from the first ALICE run are rather striking, when compared with the observations from lower energies. While a similar suppression is observed at LHC energies for peripheral collisions, when moving towards more head-on collisions --- as quantified by the increasing number of nucleons in the lead nuclei participating in the interaction --- the suppression no longer increases. Therefore, despite the higher temperatures attained in the nuclear collisions at the LHC, more $J/\psi$ mesons are detected by the ALICE experiment in Pb--Pb with respect to p--p. Such an effect is likely to be related to a regeneration process occurring at the temperature boundary between the QGP and a hot gas of hadrons (T$\approx$160~MeV).
