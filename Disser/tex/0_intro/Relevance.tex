\section*{Актуальность работы:}

Современные эксперименты в области физики высоких энергий и, особенно, столкновения релятивистских тяжелых ионов выдвигают жёсткие требования к принимаемым проектным решениям. Создаваемые установки должны быть способны измерять тонкие эффекты в присутствии высокого фона и предоставлять надежные данные для подгонки сложных многопараметрических физических моделей. Особенно жесткие требования предъявляют эксперименты с фиксированной мишенью, поскольку, за счет релятивистского буста, частицы сконцентрированы в переднем конусе, а большая плотность мишени позволяет достигать высокие частоты взаимодействий.

В эксперименте CBM, которому посвящена данная работа, при работе на ускорителе SIS100 пучки тяжелых ионов, например золота, будут разгоняться до энергии 10~\GeVperNucl и взаимодействовать с золотой фиксированной мишенью с частотой до $10^7$ ядерных взаимодействий в секунду. При этом в передний конус, ограниченный полярным углом $\SI{25}{\degree}$, будет лететь до 400~заряженных частиц в одной реакции, а максимальная угловая плотность частиц в центральной области детектора будет достигать 100~ср$^{-1}$.

Все эти факторы приводят к необходимости тщательной оптимизации конструкции установки. Для этого требуется совершенствование методов моделирования детекторов, включая реализацию высокого уровня детализации описания геометрии и возможности быстро модифицировать это описание с целью выполнения итерационных расчетов.

Также необходима разработка новых систем сбора данных, адекватных современному аппаратному обеспечению и ожидаемым потокам информации.

Кроме того, необходимы интенсивные исследования прототипов создаваемых детекторов.

В настоящей диссертации обсуждаются все три перечисленных аспекта в применении, в первую очередь, к детектору Черенковских колец эксперимента CBM (далее CBM RICH).