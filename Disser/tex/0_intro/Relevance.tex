\section*{Актуальность работы:}

Современные эксперименты в области физики высоких энергий и, особенно, столкновения релятивистских тяжелых ионов выдвигают жёсткие требования к проектным решениям при создании установок по причине высокой загрузки, высокой плотности потока частиц и высокой радиационной нагрузки, сложности и многопараметричности моделей, описывающих изучаемые эффекты, тонкости последних и высоких фонов.

Например, (рассказать c конкретными цифрами про загрузки, наблюдаемые (измеримые величины) и фоны в ряде экспериментов, в частности, flow measurements и rare probes в CBM, что-то обязательно про СТАР, АЛИСУ и какой-нибудь еще эксперимент с фиксированной мишенью, подчеркнуть особенности и жесткость требований CBM).

Все эти факторы

Требуются совершенные методы моделирования детекторов с высоким уровнем детализации и возможностью выполнения нескольких итераций расчетов, а также разработка новых систем сбора данных, адекватных современному аппаратному обеспечению и ожидаемым потокам информации.

Кроме того, необходимы интенсивные исследования прототипов создаваемых детекторов.

В настоящей диссертации обсуждаются все три перечисленных аспекта (развернуть на 1-2 абзаца) в применении, в первую очередь, к детектору Черенковских колец эксперимента CBM (далее CBM RICH).
