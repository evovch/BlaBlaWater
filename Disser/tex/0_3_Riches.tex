\section{Обзор существующих детекторов черенковских колец}\label{sec:secRiches}

%По-видимому второй рич так и остался в планах...
\subsection{COMPASS RICH-1}\label{sec:CompassRich1}

%\subsection{COMPASS}\label{sec:Compass}

% Википедия
% https://en.wikipedia.org/wiki/COMPASS_experiment

Экспериментальная установка NA58, также известная как COMPASS (``Common Muon and Proton Apparatus for Structure and Spectroscopy'') представляет собой систему из двух спектрометров длиной 60~метров за неподвижной мишенью на отводе пучка M2 ускорителя SPS в CERN. Эксперимент был предложен в 1996 году, в период с 1999 по 2001 года выполнялись работы по установке и в 2001 был выполнен первый (commissioning) запуск. Набор данных разбит на два этапа: COMPASS I (2002--2011) и COMPASS II (2012--2018). 

% http://link.springer.com/article/10.1140/epjst/e2008-00800-2

%\subsubsection{COMPASS RICH-1}\label{sec:CompassRich1}

% Самое свежее (2014) - http://compassweb.ts.infn.it/rich1/paper/tessarotto_instr14.pdf

% Ссылки
% NIM A 02 (2003) 112–116
% doi:10.1016/S0168-9002(02)02165-4
% E. Albrecht et al.
% https://wwwcompass.cern.ch/compass/detector/rich/publications/NIMA502-112.pdf

% http://wwwcompass.cern.ch/compass/proposal/pdf/proposal.pdf

% https://pub.uni-bielefeld.de/download/2301233/2301236
% 8.2.2 The Mirror Wall

Детектор Черенковских колец \mbox{RICH-1} был спроектирован в 1996~г. и функционирует с 2002~г. \mbox{RICH-1} подвергается постоянной оптиимизации, а в 2006~было выполнено обновление. Ожидается второе обновление детектора в 2016~г. \todo оно было?

Основная задача детектора Черенковских колец --- разделение $\pi$, $p$ и $K$ в диапазоне импульсов от 3~до~55~ГэВ/с в условиях высокой интенсивности (что бы это значило\todo в статье 2014 г. написано, что beam rate 40 МГц, и частота триггеров 20 кГц) в полном аксептансе спектрометра, составляющем $\pm$250~мрад по горизонтали и $\pm$180~мрад по вертикали. Для минимизации отрицательного влияния на эффективность стоящих ниже по пучку электромагнитного и адронного калориметров детектор \mbox{RICH-1} должен иметь минимум количества материала в аксептансе. Также в процессе проектирования COMPASS \mbox{RICH-1} необходимо было развивать технологии для реализации возможности регистрировать и справляться с высоким для того времени потоком данных.
\todo \textbf{По большому счёту, задачи те же, что и у CBM}

Габариты корпуса детектора COMPASS \mbox{RICH-1}, выполненного из алюминия, составляют 6.6$\times$5.3$\times$3.3~м$^3$. Внутри расположен газовый радиатор $C_{4}F_{10}$ длиной 3~м и объёмом около 83~м$^3$. Пороги по импульсу для Черенковского света: для $\pi$ --- 2.5~ГэВ/с, для $K$ --- 8.9~ГэВ/с и 17~ГэВ/с для $p$. В центре детектора проходит цилиндрический ионопровод диаметром 100~мм, наполненный гелием.
% The original beam pipe, made of 150 mm thick stainless steel, has been replaced in 2012 by a lighter pipe, made of 4 layers of metalized BoPET (25mm BoPET + 0.2mm Al): the contribution to the total material budget by the new pipe for beam particles is 0.08\% X0 (plus 0.06\% due to helium).
Газовая система закрытого типа поддерживает радиатор под избыточным давление $100\pm10$~Па.

Система фокусировки состоит из двух сферических зеркал радиусом 6.6~м, составленных из 116~сегментов шести- и пятиугольной формы, общей площадью более 21~м$^2$. Для фокусировки на фоточувствительные камеры, расположенные за пределами геометрического аксептанса, центры сфер зеркал смещены по вертикали от пучка на 1.6~м, а образовавшийся в результате этого зазор между двумя зеркалами приводит к потере 4\% площади отражающей поверхности. Зеркала были произведены компанией IMMA, Ltd., Kinskeho 703, Turnov, Czech Republic.
Коллаборацией COMPASS был разработан метод контроля индивидуальных отклонений сегментов зеркал ``на лету'' (онлайн), называемый CLAM (``a continuous line alignment and monitoring method'').

\todo \textbf{картинка - схема, оч похоже на CBM}

Исходя из необходимости иметь суммарную площадь фоточувствительных камер 5.3~м$^2$, изначально для реализации были выбраны многопроволочные пропорциональные камеры (MWPC) с сегментированным фотокатодом из CsI. \mbox{RICH-1} оборудован восемью идентичными камерами, каждая площадью 576$\times$1152~мм$^2$. Фотокатоды выполнены из двух двухсторонних печатных плат размером 576$\times$576~мм$^2$. Окна из silica-quartz состоят из двух одинаковых quartz-plates размером 600$\times$600$\times$5~мм$^3$. Сегментированный фотокатод обеспечивает размер пикселя 8$\times$8~мм$^2$ и в общей сложности 82944 канала.

% Ещё подробности тут: http://localhost/Lib/Long_term_experience_and_performance_of_COMPASS_RI.pdf

В 2006~г. с целью повышения эффективности детектора было выполнено комплексное обновление центральной области фоточувствительной камеры, составляющей 25\% от всей площади. MWPC были заменены на МА~ФЭУ с индивидуальными линзами и соответствующей считывающей электроникой. В общей сложности было установлено 4 панели по 144~МА~ФЭУ Hamamatsu R7600-03-M16, имеющими 16~каналов и входное стекло, прозрачное в ультрафиолетовой области, и специальный делитель напряжения.

% For the physics runs starting in 2016 COMPASS RICH-1 will be equipped with new MPGD-based photon detectors, which have been developed by a dedicated R&D program.

Сигнал с МА~ФЭУ считывается платами передней электроники, основанными на ASIC ``CMAD'', реализующем 8-канальный предусилитель-дискриминатор, разработанный на основе ``MAD4'' специально для COMPASS \mbox{RICH-1}. CMAD позволяет работать на частоте до 5~МГц на канал.

%The signals from the MAPMTs are read by a fast digital electronics system [36, 37] based on the 8 channels CMAD [38] preamplifier-discriminator, developed for COMPASS RICH-1 as an upgraded version in CMOS technology of the MAD4 [39] front-end chip. The CMAD has a small noise level (1 fC), the possibility to set individual channel thresholds, a good time resolution and high rate capability: it provides full efficiency up to an input rate of 5 MHz per channel. The design of the front-end boards and the optimization of the thresholds setting allows to completely suppress the MAPMTs cross talk signals while keeping the single photoelectron detection efficiency at a 95\% level.

\todo \textbf{перефразировать}
Хорошее временное разрешение МА~ФЭУ не портится за счёт использования цифровых карт DREISAM, в которых реализован ВЦП F1, имеющий временное разрешение 110~пс и может работать с чатотой до 10~МГц на канал и частоте триггера до 100~кГц.

%The good MAPMT time resolution is fully exploited with the help of digital cards, called DREISAM, housing the dead-time free F1 TDC [40] , which has a time resolution of 110 ps and can stably operate up to 10 MHz per channel input rate and 100 kHz trigger rate.

Считывающая электроника COMPASS \mbox{RICH-1} монтируется на детектор, образуя очень компактную установку, которая экранирована от внешнего электромагнитного поля медными пластинами, выполняющими также и роль радиаторов, охлаждаемых водой циркуирующей по медным трубкам.

%All the electronics components of the RICH-1 readout system are directly mounted on the detector and form a very compact setup. Each PCB is coupled to a copper plate providing both efficient electromagnetic shielding and good cooling power: thermalized water circulates in underpressure condition in thin copper pipes brazed onto the copper plates [36]. The stability and uniformity of the water cooling system has been achieved after several improvements of the distribution system and of the operation and maintenance protocols.

Оцифрованные данные с плат передней электроники передаются по оптике платам считывания CATCH, которые группируют данные и передают дальше также по оптике через S-LINK в систему сбора данных эксперимента.

%Data from the front-end cards are transferred via optical links to a set of CATCH readout-driver modules which concentrate the data and send them via S-LINK transmitter and optical fibre to the COMPASS DAQ system.

%Expected occupancy level is $\approx$5\% at a maximum trigger rate of $10^5 s^{-1}$, resulting in a maximum data flow of 2.5 GB/s. COMPASS-Gassiplex chips are used as front-end-chips. These are modified versions of the chips developed for RD26, now equipped with preamplifier, shaper and an analog-multiplexer. The intrinsic dead time is 400 nsec per event, with a peaking time of 1 usec. The value for noise is as low as 1100 electrons equivalent at a gas amplification of $\approx$6.5 mV / (fC).

%The core piece of the readout system is the total amount of 192 front-end-cards, the 60 cm long BORA boards [67], hosting the front-end chips and a first trigger level. There are 24 BORA-boards per photon chamber handling 432 analog channels. Each single BORA-board is equipped with front-end-chips, ADCs (analog digital converter), FIFOs (first in first out buffer), FPGAs (field programmable gate array) of the type VIRTEX XCV100 [68] for logic sequencers, threshold-subtraction and zero-suppression, 32-bit DSPs [69] for event packaging, on-board controls and optical links. The event processing time is 10usec. The control system for those BORA-boards is a parallel network of DSPs (digital signal processing), operated via a dedicated PC-PCI-interface: the DOLINA-boards with 8 on-board DSPs each. To avoid grounding interference between the PC and the detector all BORA-boards are optoisolated from DOLINA with the help of specific optoisolating boards. Figure 8.11 sketches the architecture of the readout system. The photon detectors reach an absolute gain of 10 4 at nominal voltage of 2000 V with photon detection efficiencies of about 75\% as presented in Figure 8.12.

\textbf{Заключение такое, что в целом конструкция этого RICH очень схожа с конструкцией CBM RICH.}

%\subsubsection{COMPASS RICH-2}\label{sec:CompassRich2}

\subsection{LHCb}\label{sec:LHCb}

% http://localhost/Lib/10.1.1.668.3561.pdf - An Overview of the Status of the LHCb RICH Detectors - 2008

\todo \textbf{ПРОСТО ЦЕЛИКОМ ПЕРЕВЕСТИ СЕКЦИЮ 2.1 ОТСЮДА:}
%http://localhost/Lib/art_10.1140_epjc_s10052-013-2431-9.pdf
Performance of the LHCb RICH detector at the LHC - 2013




LHCb is one of the four major experiments at the LHC, and is dedicated to the study of CP violation and the rare decay of heavy flavours. It is a forward spectrometer designed to accept forward-going b- and c-hadrons produced in proton-proton collisions. The layout of the spectrometer is shown in \figref{fig:LHCb}

\begin{figure}[H]
\centering
\includegraphics[width=1.0\textwidth]{pictures/LHCb.png}
\caption{Схема установки LHCb.}
\label{fig:LHCb}
\end{figure}

PID from 2 to 100~\GeVoverC
Два рича полностью охватывают необходимый угловой аксептанс 15–300 mrad with respect to the beam axis.


RICH1 covers the low and intermediate momentum region 2–40~\GeVoverC over the full spectrometer angular acceptance of 25–300 mrad. The acceptance is limited at low angle by the size of the beampipe upstream of the magnet. RICH2 covers the high-momentum region 15–100~\GeVoverC, over the angular range 15–120 mrad.

RICH1 is placed as close as possible to the interaction region. To minimize the material budget there is no separate entrance window, and the RICH1 gas enclosure is sealed directly to the exit window of the VELO vacuum tank. The downstream exit window is constructed from a low-mass carbon-fibre/foam sandwich.

RICH2 is placed downstream of the magnet, since the high momentum tracks it measures are less affected by the magnetic field. In this way it can be placed after the downstream tracking system in order to reduce material for the measurement of the charged tracks. The entrance and exit windows are again a foam sandwich construction and skinned with carbon-fibre and aluminium, respectively.

Both RICH detectors have a similar optical system, with
a tilted spherical focusing primary mirror, and a secondary
flat mirror to limit the length of the detectors along the beam
direction. Each optical system is divided into two halves on
either  side  of  the  beam  pipe,  with  RICH 1  being  divided
vertically and RICH 2 horizontally. The vertical division of
RICH 1 was necessitated by the requirements of magnetic
shielding for the photon detectors, due to their close prox-
imity to the magnet. The spherical mirrors of RICH 1 (4 seg-
ments) are constructed in four quadrants, with carbon-fibre
structure, while those of RICH 2 (56 segments), and all flat
mirrors  (16  and  40  segments  in  RICH 1  and  RICH 2  re-
spectively), are tiled from smaller mirror elements, employ-
ing a thin glass substrate.


Fluorocarbon gases at room temperature and pressure are
used as Cherenkov radiators; C4F10 in RICH 1 and CF4 in
RICH 2 were chosen for their low dispersion. The refractive
index is respectively 1.0014 and 1.0005 at 0 C, 101.325 kPa
and 400 nm. About 5\% CO2 has been added to the CF4 in
order to quench scintillation in this gas.

The momentum threshold for kaons to produce Cherenkov
light in C4F10 is 9.3 GeV/c. Particles below this momentum
would only be identified as kaons rather than pions in veto
mode, i.e. by the lack of Cherenkov light associated to the
particle. To maintain positive identification at low momen-
tum and in order to separate kaons from protons, a second
radiator is included in RICH 1: a 50 mm thick wall made of
16 tiles of silica aerogel [10] at the entrance to RICH 1. The
refractive index is n=1.03 and the light scattering length is
around 50 mm at 400 nm in pure N2. The aerogel is placed
in the C4F10 gas volume and a thin glass filter is used on the
downstream face to limit the chromatic dispersion.

...

\subsection{HERA-b RICH}\label{sec:HerabRich}

% http://localhost/Lib/rich-instr99-nim453.pdf
% 0303012.pdf - The HERA-B Ring Imaging Cerenkov Counter

Эксперимент HERA-b \todo \textbf{пару слов}. Схема экспериментальной установки представлена на \figref{fig:HERAbSetup}.

HERA-B, a fixed target experiment (see Fig. 1) at the HERA storage ring at DESY, was designed [1] to measure rare processes in the decays of B mesons. The B mesons are produced in collisions of 920 GeV/c protons with a fixed target, which consists of 8 wires which can be individually inserted into the halo of the proton beam in order not to disturb experiments measuring ep collisions. One of the essential components of the spectrometer is the Ring Imaging Cherenkov counter (RICH) [1,2,3,4]. The main purpose of the RICH counter is the identification of charged hadrons, in particular kaons from decays of B mesons. Identifying charged kaons essentially means separating them from pions in the momentum range between 3 GeV/c and about 50 GeV/c at an interaction rate of up to 40 MHz.

\begin{figure}[H]
\centering
\includegraphics[width=1.0\textwidth]{pictures/HERA_b_setup.png}
\caption{Схема установки HERA-b.}
\label{fig:HERAbSetup}
\end{figure}

Схема детектора RICH эксперимента HERA-b показана на \figref{fig:HERAbRICH}. Он имеет газовый радиатор $C_{4}F_{10}$ объёмом 108~м$^3$ и массой 1100~кг. Длина радиатора 2.8~м, коэффициент преломления n=1.00137, порог импульса для рождения Черенковских фотонов для пионов и каонов составляет 2.7~\GeVoverC и 9.6~\GeVoverC соответственно. Для частиц с $\beta=1$ Черенковский угол равен 51.5~мрад (52.4~мрад), а разница между пионами и каонами составляет 0.9~мрад при 50~\GeVoverC. Радиатор поддерживается при избыточном давлении 2.5~мбар.

Корпус детектора изготовлен из нержавеющей стали, кроме передней и задней стенок из алюминия толщиной 1~мм. После фокусировки фотоны выходят из корпуса через стенку толщиной 2~мм из оргстекла прозрачного в ультрафиолетовой области. Передняя стенка HERA-b RICH расположена на расстоянии 8.5~м от мишени.

Система фокусировки состоит из двух пар зеркал, расположенных симметрично относительно горизонтальной плоскости, проходящей через ось пучка. Первое зеркало сферическое, второе --- плоское. Сферические зеркала (см. \figref{fig:HERAbRICHmirrors}) составлены из 80 полных или обрезанных шестиугольных сегментов, имеют радиус 11.4~м и толщину 7~мм. Они покрывают прямоугольную область 6$\times$4~м, общая площадь 24~м$^2$. Каждый сегмент крепится к раме трёмя моторизированными актуаторами с удалённым управлением. Для фокусировки за пределы геометрического аксептанса сферические зеркала наклонены на 9$^\circ$ от пучка. Каждое из двух плоских зеркал состоит из 18 сегментов.

HERA-b RICH имеет две фоточувствительные камеры, расположенные соответственно над и под пучком. Одна фоточувствительная камера (см. \figref{fig:HERAbRICHcamera}) составлена из 7 супермодулей 1.1$\times$0.4~м$^2$. Поверхность камеры аппроксимирует эллиптический цилиндр. Один супермодуль состоит из $16 \times 6$ модулей, каждый экранирован от магнитного поля тонкими пластинами из soft iron. В общей сложности было установлено 1488 МА~ФЭУ R5900-00-M16 и 752 МА~ФЭУ R5900-03-M4 фирмы Hamamatsu т.е 26816 каналов. Габариты одного такого МА~ФЭУ составляют 28$\times$28~мм$^2$, а чувствительная площадь 18$\times$18~мм$^2$. Перед каждым МА~ФЭУ стоят две линзы для того, чтобы привести в соответствие площадь, занимаемую МА~ФЭУ, и чувствительную. Использование линз приводит к тому что размер пикселя становется равным $9 \times 9$~мм$^2$ у 16-пиксельного R5900-00-M16 и $18 \times 18$~мм$^2$ e 4-пиксельного R5900-03-M4.
МА~ФЭУ монтируются на платы-адаптеры $70 \times 70$~мм$^2$, на которых осуществляется распределение высокого напряжения и аттенюация сигнала с МА~ФЭУ. Платы передней электроники построены на основе чипа ASD8 --- предусилителя, формирователя и дискриминатора. Сигналы с передней электроники передаются по 16-канальному кабелю типа витая пара длиной 7.5~м к драйверам передней электроники (front end dirver, FED). 1~FED имеет 4~дочерние платы, к каждой подключено 16~кабелей, и одну материнскую. 1~FED обрабатывает 1024 канала, всего используется 28 таких наборов. Материнская плата выполняет роль интерфейса к DAQ системе всегй установки HERA-b и имеет буфер, в котором может храниться до 128 событий в ожидании сигнала от триггера первого уровня.

%33 хита на частицу с $\beta=1$.

\begin{figure}[H]
\begin{minipage}[b]{0.45\textwidth}
\includegraphics[width=0.9\textwidth]{pictures/HERAb_RICH.png}
\caption{Схема детектора HERA-b RICH.}
\label{fig:HERAbRICH}
\end{minipage}
\hspace{0.01\textwidth}
\begin{minipage}[b]{0.545\textwidth}
\includegraphics[width=1.0\textwidth]{pictures/HERAb_RICH_mirrors.png}
\caption{Схема компоновки сферических зеркал HERA-b RICH.}
\label{fig:HERAbRICHmirrors}
\includegraphics[width=1.0\textwidth]{pictures/HERAb_camera.png}
\caption{Схема заполненности одной фоточувствительной камеры HERA-b RICH. Одна ячейка соответствует модулю из 8~МА~ФЭУ.}
\label{fig:HERAbRICHcamera}
\end{minipage}
\end{figure}
