\textbf{ Во введени не надо секций. Надо научиться делать два уровня абзацев, один просто с новой строки, а другой с пропущенной строкой}

% HADES, который будет расположен выше по пучку перед CBM, когда последний будет построен.
%\todo \textbf{Этот абзац пересекается по мысли с другим абзацем который между алисой и никой. Но вводное предложение для картинки должно быть какое-то.}

%\textbf{Здесь описать то, как соотносится FAIR с другими ускорителями, на которых выполняются или планируются эксперименты аналогичные CBM. Здесь появляется фазовая диаграмма, LHC, RHIC, NICA. При этом о самих экспериментах, аналогичных CBM, разговор идёт чуть дальше после описания физической программы CBM. Может быть оставить как есть в следующей секции и ничего тут уж не писать?}

%В 2010 и 2011 была выполнена первая фаза программы скана с пучками $Au+Au$ энергий 7.7, 11.5, 19, 27 и 39~\GeVperNucl. Учитывая набранные ранее данные (62, 130 и 200~\GeVperNucl), диапазон энергий $\sqrt{s_{NN}}$, измеренных на RHIC составляет 7.7--200~\GeVperNucl. Этот диапазон энергий столкновения соответствует интервалу $\mu_{B}$ от~20 до~450~МэВ, в котором ожидается наличие перехода фазового первого рода и критической точки.

% http://iopscience.iop.org/article/10.1088/1742-6596/455/1/012037/pdf

%In 2010 and 2011 RHIC completed phase I of the Beam Energy Scan (BES) program with data sets at 7.7, 11.5, 19, 27 and 39 GeV. This is complemented by the data collected earlier at higher energies (62, 130 and 200 GeV). Together they cover the $\mu_{B}$ interval from 20 to 450 MeV, which is believed to contain the range associated with the first order phase transition and CP.

%В исходной статье есть картинка.

% https://arxiv.org/pdf/1211.1350.pdf   section 2

%\textbf{Редактировать!}

Эти измерения, несмотря на низкую статистику, позволят измерить выходы и спектры адронов и определить, опираясь на статистическую термальную модель (THERMUS~\cite{}) параметры состояния на границе химического вымораживания, называемого также адронизация.

%Ключевые характеристики ($T$, $\mu_{B}$) исследуемой области фазовой диаграммы могут быть извлечены из результатов измерений выходов адронов в столкновениях тяжёлых ионов. В первой фазе BES поперечный импульс определяется для $\pi$, $K$, $p$, $\Lambda$, $\Xi$, $K^{0}_{S}$ и $\phi$. Отношения выходов частиц используются для нахождения условий ``химического вымораживания'' (адронизация) (состояния, когда устанавливаются выходы частиц) с помощью статистической термальной модели (THERMUS).

%\textbf{Редактировать!}

% Два ключевых параметра --- температура $T_{ch}$ и барионный потенциал $\mu_{B}$ ``химического вымораживания''.

%These quantities can be extracted from the measured hadron yields. Transverse momentum spectra for the BES Phase-I energies are obtained for $\pi, K, p, \Lambda, \Xi, K^{0}_{S}$, and $\phi$. The particle ratios are used to obtain the chemical freeze-out (a state when the yields of particles get fixed) conditions using the statistical thermal model (THERMUS). The two main extracted parameters are chemical freeze-out temperature $T_{ch}$ and $\mu_{B}$. 

% https://www.epj-conferences.org/articles/epjconf/pdf/2015/14/epjconf_icnfp2014_01009.pdf
% Страница 2 - есть таблица с энергиями и количеством событий

% sonia_kabana_talk_excitedQCD_final.pdf
% Слайд 31

% \begin{table}[H]
% \caption{}
% \label{tabl:RHICenergies}
% \begin{tabular}{ | p{0.3\linewidth} | p{0.3\linewidth} | p{0.3\linewidth} | }
% \hline
% Энергия (\GeVperNucl) & Кол-во событий (млн.) & Время (недели) \\
% \hline
% 200 & 350 & 11 \\
% \hline
% 62.4 & 67 & 1.5 \\
% \hline
% 39 & 130 & 2 \\
% \hline
% 27 & 70 & 1 \\
% \hline
% 19.6 & 36 & 1.5 \\
% \hline
% 14.5 & 20 & 3 \\
% \hline
% 11.5 & 12 & 2 \\
% \hline
% 7.7 & 4 & 4 \\
% \hline
% \end{tabular}
% \end{table}

\begin{table}[H]
\caption{Наблюдаемые в области высокой плотности барионов}
\label{tabl:Experiments2}
\begin{tabular}{ | p{0.28\linewidth} | p{0.15\linewidth} | p{0.15\linewidth} | p{0.15\linewidth} | p{0.15\linewidth} | }
\hline
\textbf{Наблюдаемые} & \textbf{STAR$@$RHIC} \newline \textbf{BNL} & \textbf{NA61$@$SPS} \newline \textbf{CERN} & \textbf{MPD$@$NICA} \newline \textbf{JINR} & \textbf{CBM$@$FAIR} \newline \textbf{GSI} \\
\hline
Адроны & $+$ & $+$ & $+$ & $+$ \\
\hline
Корреляции, флуктуации \newline при высокой статистике & & & $+$ & $+$ \\
\hline
Дилептоны & & & & $+$ \\
\hline
Очарованные \newline частицы & & & & $+$ \\
\hline
\end{tabular}
\end{table}

%\begin{table}[H]
%\caption{Наблюдаемые в области высокой плотности барионов}
%\label{}
%\begin{tabular}{ | p{0.22\linewidth} | p{0.10\linewidth} | p{0.32\linewidth} | p{0.14\linewidth} | p{0.16\linewidth} | }
%\hline
%\textbf{Эксперимент} & \textbf{Адроны} & \textbf{Корреляции,} \newline \textbf{флуктуации при} \newline \textbf{высокой статистике} & \textbf{Дилептоны} & \textbf{Очарованные} \newline \textbf{частицы} \\
%\hline
%STAR$@$RHIC BNL & $+$ & & & \\
%\hline
%NA61$@$SPS CERN & $+$ & & & \\
%\hline
%MPD$@$NICA JINR & $+$ & $+$ & & \\
%\hline
%CBM$@$FAIR GSI & $+$ & $+$ & $+$ & $+$ \\
%\hline
%\end{tabular}
%\end{table}

\todo

\begin{table}[H]
\caption{Стырено у Максима}
\label{tabl:Accelerators}
\begin{tabular}{ | p{0.15\linewidth} | p{0.18\linewidth} | p{0.15\linewidth} | p{0.16\linewidth} | p{0.16\linewidth} | }
\hline
Accelerator & Laboratory & Type of experiments & Ions (heaviest) & Top energy \\
\hline
SIS 18 & GSI, Germany & fixed target & U & 2 AGeV \\
\hline
AGS & BNL, USA & fixed target & Au & 14.5 AGeV \\
\hline
SIS 100* & FAIR, Germany & fixed target & U & 11 AGeV \\
\hline
SIS 300* & FAIR, Germany & fixed target & U & 35 AGeV \\
\hline
NICA* & JINR, Russia & collider & Au & 11 AGeV \\
\hline
SPS 7 & CERN, Switzerland & fixed target & Pb & 158 AGeV \\
\hline
RHIC & BNL, USA & collider & U & 200 AGeV \\
\hline
LHC & CERN, Switzerland & collider & Pb & 5.52 ATeV \\
\hline
\end{tabular}
\end{table}

\todo

%Особое внимание в CBM уделяется легким векторным мезонам, регистрируемым не напрямую, а по продуктам распада по дилептонным каналам ($e^{+} + e^{-}$ в основном c помощью RICH и $\mu^{+} + \mu^{-}$ в основном c помощью MUCH). Также, можно, например, отметить т.н. прямые фотоны, рождаемые в первые моменты столкновения и регистрируемые электромагнитным калориметром ECAL.

\textbf{СЮДА НАДО СВОДНЫЙ НО НЕ ОЧЕНЬ ПОДРОБНЫЙ СПИСОК НАБЛЮДАЕМЫХ}

\textbf{----------------------------------------------------------------------------------------------------------}

Наблюдаемые MPD. \\
1 этап: \\
Выходы частиц и спектры \\
Флуктуации от события к событию \\
Фемтоскопия, включая $\pi$, $K$, $p$, $\Lambda$ \\
Коллективные потоки и идентификация адронов \\
Электромагнитные измерения ($e$, $\gamma$) \\

2 этап: \\
Полная множественность частиц \\
Изучение асимметрии (лучшее определение плоскости реакции) \\
Точное изучение дилептонов (расширение ECAL) \\
Экзотические частицы (мягкие фотоны, гипер-ядра) \\


Физическая программа CBM нацелена на исследование свойств сверхплотной барионной материи, образующейся в ядро-ядерных столкновениях при энергии пучка от~2~до~45~\GeVperNucl. CBM проектируется с учётом необходимости справляться с измерением высокой статистики адронных, лептонных и фотонных проб в большом аксептансе. Физическая программа включает в себя множество наблюдаемых, среди которых:

\begin{itemize}
\item выход и коллективный поток странных и очарованных адронов; \todo \textbf{ожидается что они отразят процесс становления деконфайнмента};
\item коллективный поток адронов, который особенно чувствителен к уравнению состояния ядерного вещества на ранних стадиях реакций;
\item производство частиц при пороговых энергиях (странность на SIS100 и очарование на SIS300), которое может нести важную информацию об уравнении состояний ядерной материи;
\item нестатистические отклонения от события к событию различных параметров (выходы частиц, отношения выходов), связанные с сохранением квантовых чисел (барионных, заряда, странности), которые могут служить сигналом о критической точке КХД;
\item изменение адронных масс в среде, в частности изменение, предоставляющее ценную информацию о внутренних процессах при ожидаемом восстановлении киральной симметрии в плотной барионной материи.
\end{itemize}

1) The equation-of-state of baryonic matter at neutron star densities.
The relevant measurements are:

Выход адронов в зависимости от энергии.
%The of the collective flow of hadrons which is driven by the pressure created in the early fireball (SIS100);

Выход гиперонов со странностью больше 1.

2) Свойства адронов в среде
Исследование массовых спектров лёгких векторных мезонов в среде.
Выходы и поперечный импульс (не масса) 


\textbf{CBM RICH TDR, секция 1.3}

Физическая программа CBM фокусируется на следующих пунктах:

1) Уравнение состояния барионной материи при плотностях нейтронных звёзд. \\
Соответствующие измерения: \\
1.1) Коллективный поток адронов, который обусловлен давлением в файерболе на раннем этапе (SIS100); \\
1.2) Выход частиц со странностью больше 1 в столкновениях $Au+Au$ и $C+C$ в диапазоне энергий 2--11~\GeVperNucl{} (SIS100).
% At sub-threshold energies, $\Xi$ and $\Omega$ hyperons are produced in sequential collisions involving kaons and $\Lambda$’s, and, therefore, are sensitive to the density in the fireball.

2) Свойства адронов в среде: \\
Восстановление киральной симметрии в плотной барионной среде модифицирует свойства адронов. \\
Соответствующие измерения: \\
2.1) Распределение масс векторных мезонов в среде, распадающихся в лептонные пары в столкновениях тяжёлых ионов в диапазоне энергий 2--45~\GeVperNucl{}. \\
%and for different collision systems
Лептоны являются проникающими \todo probes, несущими информацию из плотного файербола (SIS100, SIS300).
2.2) Распределения выходов и поперечных импульсов очарованных мезонов в столкновениях тяжёлых ионов как функция энергии столкновения (SIS100, SIS300).

3) Фазовый переход между адронной и партонной материями при высоких барионных плотностях. \\
Уже при энергиях SIS100 в центральных столкновениях тяжёлых ионов плотность в фаерболе превышает $\rho_{0}$ в~7~раз. Разрывность или неожиданные отклонения функции возбуждения являются чувствительными наблюдаемыми фазового перехода. \\
Соответствующие измерения: \\
3.1) Функция возбуждения выходов, спектры и коллективный поток частиц с ненулевой странностью в столкновениях тяжёлых ионов в диапазоне энергий 6--45~\GeVperNucl{} (SIS100, SIS300); \\
3.2) Функция возбуждения выходов, спектры и коллективный поток очарованных частиц в столкновениях тяжёлых ионов в диапазоне энергий 6--45~\GeVperNucl{} (SIS100, SIS300); \\
3.3) Функция возбуждения выходов и спектры лпетонных пар в столкновениях тяжёлых ионов в диапазоне энергий 6--45~\GeVperNucl{} (SIS100, SIS300);

\todo Event-by-event fluctuations of conserved quantities like baryons, strangeness, net-charge etc.

3.4) Отклонения от события к событию таких постоянных параметров, как барионы, странность, \todo
в столкновениях тяжёлых ионов как функция от энергии в диапазоне 6--45~\GeVperNucl{} (SIS100, SIS300);

4) Гиперядра, странные ди-барионы и массивные странные объекты. \\
Теоретические модели предсказывают, что в столкновениях тяжёлых ионов при энергиях SIS100 ожидается производство одиночных \todo и двойных \todo гиперядер, странных ди-барионов и тяжёлых мульти-странных короткоживущих объектов путём

5) Механизмы производства очарования, charm propagation и свойства очарованных частиц в плотной ядерной среде. \\
Соответствующие измерения: \\
5.1) Попереченое сечение и спектр импульсов частиц с открытым очарованием ($D$-мезоны) в протон-ядерных столкновениях при энергиях SIS100 и SIS300. Свойства $D$-мезонов в среде могут быть получены из соотношения $T_{A} = (\sigma_{pA} \rightarrow DX) / (A \times \sigma{pN} \rightarrow DX)$, измеренного для различных ядер мишени; \\
5.2) Попереченое сечение, спектр импульсов и коллективный поток частиц с открытым очарованием ($D$-мезоны) в ядерно-ядерных столкновениях при энергиях SIS300; \\
5.3) Попереченое сечение, спектр импульсов и коллективный поток чармония ($J/\psi$) в протон-ядерных и ядерно-ядерных столкновениях при энергиях SIS100 и SIS300.


\textbf{----------------------------------------------------------------------------------------------------------}

% http://www.fair-center.eu/for-users/experiments/nuclear-matter-physics/cbm/introduction.html

С учётом широких возможностей ускорителя (SIS100/SIS300) установка CBM изначально проектируется так, чтобы справляться с 10~МГц первичных взаимодействий для исследования очень редких наблюдаемых.

\textbf{В таблице в каждой строке указать Pb+Pb или Au+Au. Таблицу, стыренную у Максима не надо} 

В экспериментах в ЦЕРНе и Брукхейвенской национальной лаборатории поиск критической точки осуществляется только посредством регистрации спектральных характеристик потоков вторичных частиц нескольких типов, рождающихся в большом количестве. Эксперименты FAIR, благодаря высокой интенсивности первичных пучков, открывают дополнительную возможность регистрировать редкие события со сканированием обширной области фазовой диаграммы по энергиям частиц. В частности планируется впервые непосредственно исследовать признаки возникновения ``огненного шара'' (fireball) --- области ядерной материи, в которой произошёл переход от барионной фазы к кварк-глюонной фазе, --- с помощью регистрации короткоживущих векторных мезонов, распадающихся на дилептонные пары.

Диапазон энергий FAIR 2--35~\GeVperNucl{} для ионов золота хорошо подходит для проведения экспериментов в области фазовой диаграммы с высокими плотностями ядерной материи, превосходящими нормальную плотность в 8--10 раз.

Высокая интенсивность пучка и продолжительная его доступность позволят CBM впервые измерять редкие пробы, такие как очарованные адроны и лёгкие векторные мезоны (с помощью дилептонных распадов), в области энергий, предоставляемых FAIR.

Экспериментальная задача CBM --- измерять перечисленные наблюдаемые в A+A, p+A, p+p столкновениях как функцию энергии столкновения и размера системы с высокой точностью и статистикой, а также искать нарушения непрерывностей, которые могут служить сигналом о фазовом переходе первого уровня. Данная физическая программа будет выполняться измерением ядерных столкновений при экстремально высоких частотах взаимодействия.