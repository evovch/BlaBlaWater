\section{Физическая программа CBM}

\todo \textbf{Здесь упоминаются эксперименты, аналогичные CBM.}

В экспериментах в ЦЕРНе и Брукхейвенской национальной лаборатории поиск критической точки осуществляется только посредством регистрации спектральных характеристик потоков вторичных частиц нескольких типов, рождающихся в большом количестве. Эксперименты FAIR, благодаря высокой интенсивности первичных пучков, открывают дополнительную возможность регистрировать редкие события со сканированием обширной области фазовой диаграммы по энергиям частиц. В частности планируется впервые непосредственно исследовать признаки возникновения ``огненного шара'' (fireball) --- области ядерной материи, в которой произошёл переход от барионной фазы к кварк-глюонной фазе, --- с помощью регистрации корткоживущих векторных мезонов, распадающихся на дилептонные пары.

Диапазон энергий FAIR 2--35~\GeVperNucl для ионов золота хорошо подходит для проведения экспериментов в области фазовой диаграммы с высокими плотностями ядерной материи, превосходящими нормальную плотность в 8--10 раз (уже было).


% https://www.bnl.gov/npp/docs/tribble090712/Vigdor_RHIC_overview_rev2.pdf - слайд 7

RHIC планирует ``даунгрейд'' для выполнения скана по фазовой диаграмме, но \todo всё-равно не заменит FAIR.
\todo \textbf{Здесь описать то, как соотносится FAIR с другими ускорителями, на которых выполняются или планируются эксперименты аналогичные CBM. Здесь появляется фазовая диаграмма, LHC, RHIC, NICA. При этом о самих экспериментах, аналогичных CBM, разговор идёт чуть дальше после описания физической программы CBM. Может быть оставить как есть в следующей секции и ничего тут уж не писать?}


\todo \textbf{В разных источниках числа расходятся. Где-то 35, где-то 45...}

Физическая программа CBM нацелена на исследование свойств сверхплотной барионной материи, образующейся в ядро-ядерных столкновениях при энергии пучка от~2~до~45~\GeVperNucl. CBM проектируется с учётом необходимости справляться с измерением высокой статистики адронных, лептонных и фотонных проб в большом аксептансе. Физическая программа включает в себя множество наблюдаемых, среди которых:

\begin{itemize}
\item выход и коллективный поток странных и очарованных адронов; ожидается что они отразят процесс становления деконфайнмента;
\item коллективный поток адронов, который особенно чувствителен к уравнению состояния ядерного вещества на ранних стадиях реакций;
\item производство частиц при пороговых энергиях (странность на SIS100 и очарованность на SIS300), которое может нести важную информацию об уравнении состояний ядерной материи;
\item нестатистические отклонения от события к событию различных параметров (выходы частиц, отношения выходов), связанные с сохранением квантовых чисел (барионных, заряда, странности), которые могут служить сигналом о критической точке КХД;
\item изменение адронных масс в среде, в частности изменение \todo, которые предоставят ценную информацию о внутренних процессах при ожидаемом восстановлении киральной симметрии в плотной барионной материи.
\end{itemize}

\todo \textbf{плоскость реакции - ?}

Высокая интенсивность пучка и продолжительная его доступность позволят CBM впервые измерять редкие пробы, такие как очарованные адроны и лёгкие векторные мезоны (с помощью дилептонных распадов), в области энергий, предоставляемых FAIR.

Экспериментальная задача CBM --- измерять перечисленные наблюдаемые в A+A, p+A, p+p столкновениях как функцию энергии столкновения и размера системы с высокой точностью и статистикой, а также искать нарушения непрерывностей, которые могут служить сигналом о фазовом переходе первого уровня. Данная физическая программа будет выполняться измерением ядерных столкновений при экстремально высоких частотах взаимодействия.
