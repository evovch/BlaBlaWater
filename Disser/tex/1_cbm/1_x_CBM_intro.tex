Эксперимент CBM нацелен на исследование сжатой барионной материи с высокой плотностью и относительно низкой температурой с помощью редких наблюдаемых. К этим наблюдаемым относятся легкие векторные мезоны, частицы со скрытым и открытым очарованием, прецизионно измеренные анизотропии в угловых и энергетических распределениях, а в случае реализации фазового перехода первого рода --- флуктуации различных параметров от события к событию. Для реализации такой физической программы необходимо добиться рекордно высокой частоты взаимодействий. Эксперимент строится по схеме с фиксированной мишенью
с использованием максимальной интенсивности тяжёлоионных пучков, предоставляемых SIS100.
%с интенсивностью пучка до $10^9$~с$^{-1}$ и энергией, например, для золота --- до 11~\GeVperNucl{} на SIS100.
За счёт использования тонкой мишени, обеспечивающей ядерные взаимодействия с вероятностью $\approx$1\% частота последних будет достигать $10^7$~с$^{-1}$. При работе с фиксированной мишенью, большинство вторичных частиц будут лететь вперед. Отметим, что ионы, проходящие через мишень без ядерного взаимодействия, рождают большое количество дельта-электронов, дающих значительные фоны в некоторых подсистемах. Детектор оказывается в довольно жестких условиях эксплуатации. С одной стороны, имеют место высокие градиенты угловой плотности частиц, с другой стороны, наблюдаются высокие частоты взаимодействий. Таким образом, детектор должен быть спроектирован с учетом требований переменной гранулярности, высокой радиационной стойкости и способности обрабатывать большой поток данных. Последнее требование достигается путем использования самозапускающейся электроники. В таком подходе каждый канал электроники вырабатывает сообщение при условии преодоления сигналом некоторого порога. Сообщение содержит в общем случае идентификатор сработавшего канала, временную отметку и амплитудную информацию. После срабатывания каналы на некоторое время теряют чувствительность, а остальные каналы продолжают ожидать следующее событие. В результате при регистрации каждого события имеется некоторое количество случайно распределенных нечувствительных каналов, что приводит к необходимости устойчивости алгоритмов реконструкции к пропущенным хитам. Это касается как треков частиц в различных комбинациях детекторов, так и черенковских колец.

Указанные жёсткие требования к детектору приводят к необходимости тщательной оптимизации множества конструктивных элементов, компоновочных решений и алгоритмов обработки данных. Процедура оптимизации включает в себя создание и многократную передачу детальной геометрической информации между САПР и программными пакетами, моделирующими разнообразные физические свойства и эффекты от механических и тепловых до связанных с прохождением частиц через вещество. Создание на основе инженерной модели детальной геометрии для Монте-Карло пакетов GEANT4 и CbmRoot не имеет автоматизированного решения. В ходе работы над данной диссертацией был разработан (глава~2) и применен (глава~3) инструментарий для обмена геометрической информацией ``CATIA-GDML geometry builder''. САПР CATIA~v5 была выбрана по той причине, что именно она является основной в таких научных центрах как ЦЕРН и ГСИ. Во многих других центрах, например ОИЯИ, данная САПР является одной из официально поддерживаемых.

Концепция бестриггерного сбора данных является относительно новой, и опыт её применения в крупных экспериментах пока что отсутствует. В связи с этим особую важность приобретает испытание и исследование свойств прототипов таких систем. В главе~4 описывается оборудование и программное обеспечение, необходимое для полнофункционального прототипа детктора черенковских колец эксперимента CBM. В главе~5 с помощью данного прототипа исследуются свойства системы считывания на основе МА~ФЭУ H12700 и влияние этих свойств на характеристики детектора CBM RICH.

% \todo Где-то в конце главы 1 надо резюмировать более конкретно темы и мотивы для глав 2-5.

%1.3.5 минимизация магнитного поля в области фотодетекторов
%Измеренеия, выполненные как группой CBM RICH, так и авторами работы [] показывают, что МА~ФЭУ H12700 сохраняют работосопособность с %приемлемой эффективностью регистрации одиночных фотоэлектронов в магнитных полях до 2~мТл.

Ниже обсуждаются особенности компоновки и конструктивных решений различных подсистем эксперимента.

\bigskip
