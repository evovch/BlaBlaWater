\bigskip

Эксперимент CBM нацелен на исследование сжатой барионной материи с высокой плотностью и относительно низкой температурой с помощью редких наблюдаемых. К этим наблюдаемым относятся легкие векторные мезоны, частицы со скрытым и открытым очарованием, прецизионно измеренные анизотропии в угловых и энергетических распределениях, а в случае реализации фазового перехода первого рода --- флуктуации различных параметров от события к событию. Для реализации такой физической программы необходимо добиться рекордно высокой частоты взаимодействий. Эксперимент строится по схеме с фиксированной мишенью с интенсивностью пучка до $10^9$~с$^{-1}$ и энергией, например, для золота --- до 11~\GeVperNucl{} на SIS100. За счёт использования тонкой мишени, обеспечивающей ядерные взаимодействия с вероятностью $\approx$1\% частота последних будет достигать $10^7$~с$^{-1}$. Благодаря работе с фиксированной мишенью, большинство вторичных частиц будут лететь вперед. Отметим, что ионы, проходящие через мишень без ядерного взаимодействия, рождают большое количество дельта-электронов, дающих значительные фоны в некоторых подсистемах. Детектор оказывается в довольно жестких условиях эксплуатации. С одной стороны, имеют место высокие градиенты угловой плотности частиц, с другой стороны, наблюдаются высокие частоты взаимодействий. Таким образом, детектор должен быть спроектирован с учетом требований переменной гранулярности, высокой радиационной стойкости и способности обрабатывать большой поток данных. Последнее требование достигается путем использования самозапускающейся электроники. В таком подходе каждый канал электроники вырабатывает сообщение при условии преодоления сигналом некоторого порога. Сообщение содержит в общем случае идентификатор сработавшего канала, временную отметку и амплитудную информацию. После срабатывания каналы на некоторое время теряют чувствительность, а остальные каналы продолжают ожидать следующее событие. В результате при регистрации каждого события имеется некоторое количество случайно распределенных нечувствительных каналов, что приводит к необходимости устойчивости алгоритмов реконструкции к пропущенным хитам. Это касается как треков частиц в различных комбинациях детекторов, так и черенковских колец. Ниже обсуждаются особенности компоновки и конструктивных решений различных подсистем эксперимента.