\section{Некоторые задачи, связанные с разработкой детектора RICH эксперимента CBM}

% М.б. убрать по ключевому слову "главе" ссылки на последующие главы выше по тексту.

% Выполняемая с применением
Важным аспектом разработки CBM RICH является тщательная оптимизация всех основных систем, что включает в себя как разнообразное моделирование, так и исследование прототипов. Технические решения, касающиеся механических конструкций, систем фокусировки, регистрации черенковских фотонов и газообеспечения должны быть выверены с точки зрения обеспечения требуемых физических характеристик прибора, технологичности, стабильности работы, влияния на детекторы, стоящие ниже по пучку.
% --------
Моделирование включает в себя как метод КЭ для механических свойств и магнитных полей, так и Монте-Карло моделирования прохождения частиц для исследования физических характеристик и влияния на нижестоящие детекторы.
% --------
Глава~2 данной диссертации посвящена развитию методов многократного и детального обмена геометрической информацией между конструкторскими САПР и пакетами МК моделирования, а глава~3 --- применению этих методов для оптимизации систем фокусировки, механических конструкций, магнитных экранов, ?систем? регистрации черенковских фотонов CBM RICH и некоторым другим случаям применения разработанной методики.
% 

% исследовать в тестах?
Для достижения наилучших физических характеристик CBM RICH необходимо собрать и исследовать прототип системы считывания и сбора данных как в пучковых тестах в составе полнофункционалного прототипа детектора черенковских колец, так и в лабораторных условиях.

Описание созданных установок и разработанного ПО содержится в главе~4.
В главе~5 исследуются вопросы, связанные с
влиянием особенностей системы считывания на эффективность регистрации одиночных фотоэлектронов,
временными свойствами системы считывания и
возможностью повышения эффективности за счет использования сместителя спектра.
% эффективности ?чего?