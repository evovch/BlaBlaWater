\section*{Заключение}\label{sec:secSummary}
\addcontentsline{toc}{section}{Заключение}

Исследованы свойства прототипа системы считывания и сбора данных детектора RICH эксперимента CBM. Подробно охарактеризован 64-канальный модуль, состоящий из МА~ФЭУ H12700, четырёх плат предусилителей-дискриминаторов PADIWA и одной платы TRB~v3, выполняющей функции ВЦП и концентратора данных. Описаны необходимые для работы прототипа модули ПО. Продемонстрировано, что ВЦП имеют временное разрешение 21~пс~(FWHM) при использовании калибровки точного времени. Применение поканальной кусочно-линейной псевдо-калибровки ухудшает временное разрешение до 50~пс~(FWHM), а единой для всех каналов усреднённой псевдо-калибровки приводит к значению временного разрешения 64~пс~(FWHM) в наиболее неблагоприятных случаях. Обсуждена процедура калибровки задержек между каналами а также стабильность полученных задержек. Дрейф задержек не превышает 0.5~нс за все время измерений (??часов).
% ? Начало следующего предложения.
Рассмотрена возможность использования спектров ``времени над порогом''~(ToT) для отбора корректных хитов и коррекции временной привязки. Выявлено, что спектр ToT имеет многопиковую структуру по причине периодических наводок. Это препятствует использованию этого параметра в анализе. Выявленные схемотехнические недостатки будут устранены в следующей версии плат считывающей электроники. Исследованы временные свойства сместителя спектра и его влияние на эффективность регистрации черенковских колец. Наиболее интенсивная быстрая компонента характеризуется временем высвечивания 1.1~нс, но имеются также компоненты с характерными временами 3.8~нс и 45~нс. Проведено сравнение медленного аналогового и быстрого временного считывания МА~ФЭУ. Выявлено проявление особенностей одноэлектронного спектра в том, как эффективность регистрации фотоэлектронов и вероятность появления ложных хитов зависят от порога дискриминатора. Исследовано временное разрешение всего канала считывания для различных по величине множеств каналов: от одной пары до 256~штук. Наихудшее из полученных значений составляет 1.2~нс, что определяется в первую очередь отсутствием коррекции временной отметки в зависимости от амплитуды сигнала и дрейфом задержек между каналами. Полученные результаты достаточны для использования исследованной схемы считывания и сбора данных в эксперименте CBM, однако устранение выявленных недостатков позволит создать запас по эффективности и повысить надежность системы при долговременной эксплуатации.
