\section{Особенности МА~ФЭУ H12700}\label{section:secMapmt}

Многоанодный фотоэлектронный умножитель (МА~ФЭУ) H12700 фирмы Hamamatsu~\cite{H12700MANUAL}, появившийся на рынке в~2013~г., подробно охарактеризован в работах~\cite{CALVI, CALVI2}. Он обладает следующими достоинствами: большая доля площади поперечного сечения, приходящаяся на светочувствительные пиксели, квадратная форма, что позволяет перекрывать без потерь значительные площади (плотность упаковки~87\%), малое время прохождения однофотоэлектронного сигнала через динодную систему, малый разброс этого времени от события к событию, низкие перекрёстные помехи и низкая скорость счета тепловых электронов. Некоторые свойства данного прибора показаны в табл.~\ref{tabl:MAPMT}, по большинству параметров он превосходит своего предшественника МА~ФЭУ H8500~\cite{H8500MANUAL}.

\begin{table}[H]
\caption{Свойства МА~ФЭУ H12700B-03.}
\label{tabl:MAPMT}

%\begin{tabular}{ | p{0.25\linewidth} | p{0.25\linewidth} | p{0.25\linewidth} | p{0.25\linewidth} | }
%	\hline
%	Темновой счёт на канал, Гц & Темновой счёт на весь МА~ФЭУ, кГц & Время нарастания сигнала, нс & Разброс времени развития электронной лавины, нс\\
%	\hline
%	$ \approx $ 10 & <1,0 & 0,64 & 0,28\\
%	\hline
%\end{tabular}

\begin{tabular}{ | p{0.7\linewidth} | p{0.1\linewidth} |}
	\hline
	Темновой счёт на канал, Гц & $ \approx $ 10 \\
	\hline
	Темновой счёт на весь МА~ФЭУ, кГц & <1,0 \\
	\hline
	Время нарастания сигнала, нс & 0,64 \\
	\hline
	Разброс времени развития электронной лавины, нс & 0,28 \\
	\hline
\end{tabular}

\end{table}

Данный МА~ФЭУ имеет двухщелочной фотокатод. Спектральная чувствительность МА~ФЭУ в версии H12700B-03, используемой в настоящей работе, определяется входным окном, сделанным из стекла, прозрачного в ультрафиолетовой области. Коротковолновая граница спектра чувствительности $ \lambda_{min} $=185~нм, а максимум квантовой эффективности составляет 33\% и достигается при длине волны $ \lambda $=380~нм. Такие спектральные характеристики хорошо подходят для регистрации черенковского излучения, лежащего в ультрафиолетовой области. Каждому аноду соответствует канал МА~ФЭУ, состоящий из своего фрагмента динодной системы и области фотокатода, называемой пикселем. Среднеквадратичное отклонение коэффициентов усиления в каналах МА~ФЭУ от среднего значения не превышает 16\%~\cite{H12700MANUAL}. Разброс квантовой эффективности между пикселями по нашим данным составляет $ \pm $10\%.

Имеются исследования~\cite{MAPMTRADHARD, THECBMRICHPROJ16, THECBMRICHDET16}, показывающие, что радиационная стойкость прибора достаточна для использования в эксперименте~CBM. Также продемонстрирована работоспособность прибора в магнитном поле до 2,5~мТл~\cite{CALVI} без значительного падения характеристик. Использование магнитных экранов и выбор оптимального расположения фотодетектора в пространстве делают этот МА~ФЭУ пригодным для использования в эксперименте~CBM. Отметим, что к этому прибору проявляют интерес и другие эксперименты, например, он рассматривается для обновления LHCb~\cite{CALVI}.

Наряду с перечисленными достоинствами, МА~ФЭУ H12700B-03 имеет некоторые особенности, не имеющие аналогов в традиционных МА~ФЭУ и требующие особого внимания при реализации канала считывания. Размножение электронов в динодной системе происходит в одном и том же вакуумном объеме для всех каналов. Помещённая в единый вакуумный объём динодная система типа ``Metal Channel'', см.~\figref{fig:MetalChannel}~\cite{MCdynodeSys}, отличается тем, что она довольно компактна, едина для всех каналов и позволяет добиться отличных временных свойств. Электронные лавины, соответствующие разным каналам, отличаются местом прохождения через динодную систему. Имеют место такие эффекты как выбивание электронов из динодов фотонами, прошедшими сквозь фотокатод, и отклонение электронов от идеальной траектории за счет разброса энергий. Последняя особенность приводит к попаданию электронов на последующие стадии динодной системы, минуя предыдущие, и перетеканию всей или части электронной лавины в соседний канал. Перетекание части лавины в соседний канал имеет место в более чем 25\% случаев при равномерном освещении всего фотокатода. Величина перетекающего заряда составляет от~3\% до~7\% в зависимости от взаимного расположения каналов МА~ФЭУ~\cite{CALVI}. Вероятность того, что лавина от фотоэлектрона полностью разовьётся в соседнем канале зависит от взаимного расположения каналов и составляет при равномерном освещении от~0,1\% до~2\%~\cite{KOPFERDISS}. Кроме того, при наличии относительно большого сигнала в одном из каналов, наблюдается биполярная наводка в каналах, имеющих диноды в одном ряду. При интегрировании этой наводки возможно формирование низкоамплитудных импульсов в нескольких каналах. В классическом МА~ФЭУ такие эффекты не наблюдаются из-за отсутствия связи с соседними каналами, наличия развитой системы фокусировки и такой конструкции динодной системы, что диноды имеют большую площадь и последующие стадии полностью экранируются предыдущими.

\begin{figure}[H]
\centering
\includegraphics[width=0.7\textwidth]{pictures/2_Metal_channel.png}
\caption{Схема динодной системы типа ``Metal Channel''.}
\label{fig:MetalChannel}
\end{figure}

Описанные особенности приводят к формированию в одноэлектронном спектре низкоамплитудной части, сливающейся с шумами и отделенной от основного пика довольно глубокой ложбинкой. Проявления этого эффекта в наших измерениях обсуждаются в секции~\ref{section:secNxVsPadiwa}.
