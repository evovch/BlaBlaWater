\section*{Аннотация}\label{sec:secAbstract}
\addcontentsline{toc}{section}{Аннотация}

%\begin{abstract}

Подробно охарактеризован 64-канальный модуль считывания и приема данных, состоящий из МА~ФЭУ H12700, четырех плат предусилителей-дискриминаторов PADIWA и одной платы TRB~v3, выполняющей функции ВЦП и концентратора данных. Описаны необходимые для работы прототипа модули ПО. Продемонстрировано, что ВЦП имеют временное разрешение от~21 до~64~пс~(FWHM) в зависимости от способа калибровки точного времени. Проведена калибровка задержек между каналами. Дрейф задержек не превышает 0.5~нс за все время измерений (??часов). Исследованы спектры ``времени над порогом''~(ToT).
%ВыявленЫ?
Выявлено влияние периодических наводок и необходимость совершенствования схемотехнических решений. Исследованы временные свойства сместителя спектра и его влияние на эффективность регистрации черенковских колец. Наиболее интенсивная компонента характеризуется временем высвечивания 1.1~нс, также имеются компоненты с характерными временами 3.8~нс и 45~нс. Выявлено влияние особенностей одноэлектронного спектра на эффективность регистрации фотоэлектронов и вероятность появления ложных хитов. Временное разрешение совокупности из 256~каналов составляет 0.9 (?) нс. Полученные результаты достаточны для использования исследованной схемы считывания и сбора данных в эксперименте CBM, однако устранение выявленных недостатков позволит создать запас по эффективности и повысить надежность системы при долговременной эксплуатации.

%\end{abstract}